%%=============================================================================
%% Literatuurstudie
%%=============================================================================

\chapter{Literatuurstudie}
\label{ch:literatuurstudie}

\section{De IBM Z Systems omgeving}
\label{sec:De IBM Z Systems omgeving}
Het is belangrijk om te weten wat een mainframe is en wat de hoofdpunten van de technologie zijn. Het is moeilijk om een goede definitie te plakken op deze term maar het is ontwikkelt door IBM en het wordt vooral gebruikt door grote bedrijven om belangrijke applicaties te hosten of veel transacties te kunnen doorvoeren. Hoewel dit ook mogelijk is op een kleinschalige server, zou het resultaat niet hetzelfde zijn omdat Mainframes miljarden transacties per dag zou kunnen uitvoeren zonder enige vertraging. \autocite{BasuMallick2023} \\
De Mainframe wordt vaak in vergelijking gebracht met een traditionele server die je vind in een datacenter. Deze vergelijking is niet onterecht omdat ze wel een gelijkaardige functie hebben. Een Mainframe zoals de hedendaagse IBM z16 heeft de mogelijkheid 19 miljard transacties per dag uit te voeren, wat ook verklaard waarom deze systemen gebruikt worden door 92 van de top 100 banken ter wereld. De z16 heeft ook 40 terabyte werkgeheugen wat 1200 keer het aantal is in hedendaagse high-performance computers. \autocite{Tozzi2022} \\ 
De z16 is ook `Quantum safe`: de bedreiging van Quantum computers in de toekomst blijft groeien en er is nog geen directe oplossing voor. IBM heeft hiervoor geïnvesteerd in Crypto Express 8S hardware security modules om data op de Mainframe te beschermen en Quantum safe te maken. De nieuwe modules bevatten nieuwe quantum safe encryptie algoritmes die geëvalueerd zijn door de US National Institute of Standards and Technology. \autocite{Sayer2022}
\\
De software die wordt gebruikt is ook zeer verschillend. Hoewel er verschillende besturingssystemen mogelijk zijn, is z/OS wat ontwikkelt is door IBM nog steeds het meest gebruikt. Dit vooral omdat dit het meest compatibel is met de hardware componenten en het blijft ondersteund door IBM zelf. Dit bevat tools zoals Time Sharing Option (TSO), Interactive System Productivity Facility (ISPF) en System Display qnd Search Facility (SDSF) om er een paar op te noemen. z/OS heeft ook een andere manier om data op te slaan door middel van datasets. Hier bestaan verschillende 4 verschillende versies van: Sequentieel, Pqrtitioneel, 

\section{IBM Developer for z/OS}
\label{sec:IBM Developer for z/OS (IDz)}
De testomgeving waarin we zullen werken is IBM Developer for z/OS (IDz) versie 16.0.2. Dit programma is volgens de definitie van Joy Spohn een toolset voor het ontwikkelen en opzetten van hybride cloud applicaties op z/OS. \textcite{Spohn2023}. 