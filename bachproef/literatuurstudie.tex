%%=============================================================================
%% Literatuurstudie
%%=============================================================================

\chapter{Literatuurstudie}
\label{ch:literatuurstudie}

\section{De IBM Z Systems omgeving}
\label{sec:De IBM Z Systems omgeving}
Het is belangrijk om te weten wat een mainframe is en wat de hoofdpunten van de technologie zijn. Het is moeilijk om een goede definitie te plakken op deze term maar het is ontwikkelt door IBM en het wordt vooral gebruikt door grote bedrijven om belangrijke applicaties te hosten of veel transacties te kunnen doorvoeren. Hoewel dit ook mogelijk is op een kleinschalige server, zou het resultaat niet hetzelfde zijn omdat Mainframes miljarden transacties per dag zou kunnen uitvoeren zonder enige vertraging. \autocite{BasuMallick2023} \\

De Mainframe wordt vaak in vergelijking gebracht met een traditionele server die je vind in een datacenter. Deze vergelijking is niet onterecht omdat ze wel een gelijkaardige functie hebben. Een Mainframe zoals de hedendaagse IBM z16 heeft de mogelijkheid 19 miljard transacties per dag uit te voeren, wat ook verklaard waarom deze systemen gebruikt worden door 92 van de top 100 banken ter wereld. De z16 heeft ook 40 terabyte werkgeheugen wat 1200 keer het aantal is in hedendaagse high-performance computers. \autocite{Tozzi2022} \\

De z16 is ook `Quantum safe`: de bedreiging van Quantum computers in de toekomst blijft groeien en er is nog geen directe oplossing voor. IBM heeft hiervoor geïnvesteerd in Crypto Express 8S hardware security modules om data op de Mainframe te beschermen en Quantum safe te maken. De nieuwe modules bevatten nieuwe quantum safe encryptie algoritmes die geëvalueerd zijn door de US National Institute of Standards and Technology. \autocite{Sayer2022} \\



\subsection{Het hoofdbesturingssysteem z/OS}
Een IBM Mainframe ondersteund meerdere besturingssystemen maar de meest gebruikte is z/OS. Dit is, samen met de IBM Z Systems, ontwikkeld door IBM waardoor dit het meest compatibel is met de hardware componenten en het blijft ondersteund door IBM zelf. Deze heeft verschillende karakteristieken zoals workload manager (WLM) om het uitvoeren van jobs te plannen. Dit besturingssysteem kan gezien worden als hybride omdat het moderne taken van andere besturingssystemen neemt en combineerd met de architectuur van een IBM Mainframe. Het heeft ook de mogelijkheid om terug te draaien naar een vorige versie zonder dat dit problemen zal veroorzaken met het systeem. \autocite{Rupp2022}

\subsection{Software in z/OS}
Dit besturingssysteem bevat tools zoals TSO, ISPF en SDSF om er een paar op te noemen. \\

TSO staat voor Time Sharing Option en is in principe de command line interface op de mainframe. Dit laat meerdere gebruikers toe om een interactieve sessie op te starten met z/OS via hun eigen inloggegevens. Zoals een traditionele CLI bestaat dit uit een prompt waar je commando's kunt ingeven om acties uit te voeren op het systeem. In TSO wordt dit een `READY` prompt genoemd omdat het een READY melding geeft als je commando's kunt invoeren. \autocite{IBM} \\

Hoewel TSO vrij krachtig is, zal dit door de meeste eindgebruikers gebruikt worden in combinatie met ISPF of \textit{Interactive System Productivity Facility}. Dit is een GUI dat bestaat uit menu's en panelen die allemaal verschillende functies kunnen uitvoeren \autocite{IBM}. Veel gebruikte functies zijn het aanmaken en schrijven van COBOL of PL/1 programma's. ISPF biedt nog meer verschillende functies zoals het aanmaken van datasets tot een connectie maken met de database op de mainframe. Hierin kun je dus eigenlijk alles doen wat het platform te bieden heeft maar wel in de lijnen van de rechten die je hebt als gebruiker. \\

System Display and Search Facility of SDSF is volgens de \textcite{IBM2023} documentatie een interface om jobs en hun output te kunnen zien. Zo is er ook de mogelijkheid om een job te doen stoppen, bijhouden of vrijgeven. Met bijhouden bedoelen we niet laten uitvoeren tot iemand een teken geef dat het uitgevoerd mag worden. Vrijgeven wordt gebruikt om de fysieke middelen zoals CPU percentage vrij te geven zodat een andere job deze kan gebruiken. \\ 
Het biedt ook informatie over het z/OS systeem zodat u dit kan monitoren, managen en controlleren. Hoewel het vooral gebruikt wordt om de status van een uitgevoerde job te zien, wordt deze tool ook gebruikt om fysieke toestellen te controleren (zoals een printer), de fysieke middelen zoals CPU of geheugen beheren en de system log en messages bekijken.

\subsection{Datasets}
z/OS heeft ook een andere manier om data op te slaan door middel van datasets. Dit is volgens de documentatie van \textcite{IBM} een collectie van gerelateerde data records dat opgeslagen en opgehaald wordt door een toegewezen naam. Dit zou u kunnen zien als een bestand in andere besturingssystemen zoals in Windows of Linux. \\

Hier bestaan er verschillende soorten van:

\begin{itemize}
    \item Sequentiële dataset (Seq)
    \item Partitionele dataset (Pds)
    \item Virtual storage access method (VSAM)
\end{itemize} \\

Sequentiële datasets bevatten eigenlijk records die achter elkaar opgeslagen zijn. Dit heeft als nadeel dat als je bijvoorbeeld record 20 wilt lezen, je eerst voorbij de voorgaande 19 records moet gaan. Dit is vooral nuttig om grote hoeveelheden data in op te slaan. \\

Een partitionele dataset is meer voor programma's in te schrijven. Deze dataset bevat allemaal `members` die op hun beurt dan de effectieve data bevatten. Dit heeft als voordeel dat je in een pds de members kunt aanspreken in een willekeurige volgorde. Deze vorm van een dataset wordt ook wel een library genoemd. \\

Een VSAM biedt een complexere manier van toegang tot verschillende soorten data en is vooral bedoeld voor applicaties. Door hun complexiteit kunnen ze niet frequent bekeken of aangepast worden in ISPF in tegenstelling tot een Pds. Een VSAM kun je verdelen in 4 verschillende datasets:
\begin{itemize}
    \item Key Sequence Data Set (KSDS): \\Dit is het meest voorkomend en slaat data op op basis van een key, value systeem
    \item Entry Sequence Data Set (ESDS): \\Dit houdt de records in een sequentiële volgorde bij en worden ook gelezen in deze volgorde. Dit is vooral gebruikt door IMS, DB2 en z/OS Unix.
    \item Relative Record Data Set (RRDS): \\Dit houdt records bij die je kunt ophalen op basis van een nummer. Zo heb je toegang tot records die opgeslagen zijn op plaats 100 zonder dat je door de eerste 99 records moet gaan. Dit is te vergelijken met een KSDS.
    \item Lineair Data Set (LDS): \\Dit houdt data bij in een byte stream en is de enige vorm van dit soort in een traditionele z/OS file
\end{itemize}

\subsection{Unix op een mainframe}
IBM heeft veel ingezet op modernisatie. Zo is er een Unix Systems Services (USS) omgeving bijgekomen die geïntegreerd is in het traditionele besturingssysteem z/OS. Er is ook een volledige Mainframe die enkel Linux heeft als besturingssysteem namelijk de LinuxOne. In dit onderzoek zal er enkel gekeken worden naar een z15 die een USS omgeving heeft. \\

Omdat de USS dus samenwerkt met z/OS heb je veel meer functies die je kunt gebruiken. Zo kun is er de mogelijkheid op XML parsing, OpenSSH, de IBM HTTP Server for z/OS, de z/OS SDK for Java en nog veel meer. \autocite{Dhawan2013} \\
 
Dit systeem biedt een hierarchische bestandssysteem (HFS) samen met een zSeries bestandssysteem (zFS). \autocite{Precisely2020} De HFS is wel bekend voor de meeste UNIX gebruikers: Dit is een hierarchie van directories met bestanden of subdirectories die grafisch weergegeven kunnen worden in een tree view \autocite{HCLTechnologies2022}. \\ 
zFS is iets minder bekend en kan gebruikt worden in pllaats van of als toevoeging op het traditionele HFS. Dit heeft vooral zijn waarde door zijn sterke performantie in bestanden die vaak worden gebruikt. Het verminderd ook het risico van het verlies in updates omdat het data asynchroon schrijft in plaats van te wachten op een sync interval. Bestanden in dit systeem kunnen aangepast worden door middel van een Application Programming Interface (API) en kunnen zelf in de HFS toegevoegd worden zonder enige problemen. \autocite{IBM2012} \\

Unix bestanden op de mainframe worden bijna op dezelfde manier gebruikt als op een traditioneel unix systeem. Het kan een Java, C++ of Python programma's bevatten. Deze programma's kunnen ook bestanden schrijven of schrijven in bijvoorbeeld een JSON of YAML formaat. Die kunnen op hun beurt dan gebruikt worden om analyses te doen op bepaalde data. Het hangt dus allemaal af van de use case om te zien op welke manier deze unix bestanden gebruikt worden. \autocite{Precisely2020}

\subsection{Andere besturingssystemen}
Zoals eerder vermeld is z/OS niet het enige besturingssysteem dat beschikbaar is op de mainframe. Hoewel dit door IBM aangeboden wordt, zijn er andere mogelijkheden die elk hun eigen sterktes en zwaktes hebben. \\

\begin{itemize}
    \item z/Virtual Machine (z/VM)\\
    Dit is een hypervisor type 1 en kan gebruikt worden om meerdere besturingssystemen in te hosten. Dit bestaat uit een control program (cp) en een conversation monitoring system (CMS). De cp is verantwoordelijk voor het creëren van meerdere virtuele machines op basis van de fysieke hardware middelen. Het zorgt ook voor data en applicatie beveiliging voor alle systemen die in het systeem zitten. De CMS zit in een eigen virtuele machine biedt een interactieve sessie tussen de andere virtuele machines en de eindgebruikers. \autocite{IBMb} \\
    
    \item z/Virual Storage Extended (z/VSE) \\
    Dit besturingssysteem is vooral nuttig voor kleinere bedrijven die geen complexe batch -of transactie jobs moeten processen. Het is mogelijk dat naarmate ze groeien, ze overgaan naar z/OS als z/VSE niet genoeg blijkt te zijn. Het design van dit besturingssysteem maakt het perfect voor meer routine batch jobs parallel uit te voeren. Meestal wordt z/VM ook gebruikt als een teminal interface voor development en systeembeheer in z/VSE. \autocite{IBMb} \\
    
    \item Linux for System z \\
    
    \item z/Transaction Processing Facility (z/TPF) \\
\end{itemize}


\section{IBM Developer for z/OS}
\label{sec:IBM Developer for z/OS (IDz)}
De testomgeving waarin we zullen werken is IBM Developer for z/OS (IDz) versie 16.0.2. Dit programma is volgens de definitie van \textcite{Spohn2023} een toolset voor het ontwikkelen en opzetten van hybride cloud applicaties op z/OS. \\
Het is een eclipse based programma met de mogelijkheid hebt om een connectie te maken met verschillende omgevingen op de mainframe (bv de USS of z/OS omgeving). Zo kunt u alle bestanden zien, openen en aanpassen. Omdat de data nogsteeds op de mainframe staat, kunnen wijzigingen direct gezien worden ookal bekijkt u het in een ander programma. Als u bijvoorbeeld een bestand wijzigt via IDz, kunt u de wijzigingen direct zien in ISPF. \\
Dit programma is vooral ontwikkelt om een envoudigere IDE te bieden om in te programmeren aangezien niet iedereen direct bekend is met ISPF. \\

Zoals vermeld zal er gebruik worden gemaakt van IDz versie 16.0.2 . In het overzicht van \textcite{IBM2024}, ondersteund deze versie syntax veranderingen voor COBOL 6.4, PL/1 6.1 en REXX. Er is ook een ZUnit update wat vooral het gebruik van deze tool makkelijker maakt. \\

ZUnit staat voor z/OS Automated Unit Testing en is een framework dat gebruikt wordt in IDz om COBOL en PL/1 programma's te testen. Hiervoor maakt het gebruik van verschillende `samples` die door IBM zijn ontworpen (bv. Enterprise COBOL CALL02.cbl sample test case). Het gebruik van deze tool maakt het makkelijker voor developers om hun code (geschreven in COBOL of Pl/1) te testen op de mainframe. Dit maakt het schrijven van code in IDz veel efficiënter. \autocite{IBM2024a}

\subsection{Wizards}

\subsection{Open source}


\section{IBM Z in een bankomgeving}
Dit onderzoek wordt uitgevoerd in een bankomgeving dus is het wel interessant om te bekijken welke voordelen deze technologie heeft in zo een omgeving. \\ 
Volgens \textcite{Turner2022} gebruiken de meeste banken een IBM mainframe omdat ze de rekenkracht kunnen bieden die banken nodig hebben om efficiënt te kunnen werken. Kenmerken zoals robuustheid, betrouwbaarheid en snelle processing kracht spelen ook een grote rol aangezien het van groot belang is dat het systeem bijna altijd actief moeten zijn. De mainframe toont hier zijn sterkte door de 8 nines oftwel 99,999999\% van de tijd beschikbaar per jaar. \autocite{IBMa}