%%=============================================================================
%% Literatuurstudie
%%=============================================================================

\chapter{Literatuurstudie}
\label{ch:literatuurstudie}

\section{De IBM Z Systems omgeving}
\label{sec:De IBM Z Systems omgeving}
Het is belangrijk om te weten wat een mainframe is en wat de hoofdpunten van de technologie zijn. Het is moeilijk om een goede definitie te plakken op deze term maar het is ontwikkelt door IBM en het wordt vooral gebruikt door grote bedrijven om belangrijke applicaties te hosten of veel transacties te kunnen doorvoeren. Hoewel dit ook mogelijk is op een kleinschalige server, zou het resultaat niet hetzelfde zijn omdat Mainframes miljarden transacties per dag zou kunnen uitvoeren zonder enige vertraging. \autocite{BasuMallick2023} \\ \\
De Mainframe wordt vaak in vergelijking gebracht met een traditionele server die je vind in een datacenter. Deze vergelijking is niet onterecht omdat ze wel een gelijkaardige functie hebben. Een Mainframe zoals de hedendaagse IBM z16 heeft de mogelijkheid 19 miljard transacties per dag uit te voeren, wat ook verklaard waarom deze systemen gebruikt worden door 92 van de top 100 banken ter wereld. De z16 heeft ook 40 terabyte werkgeheugen wat 1200 keer het aantal is in hedendaagse high-performance computers. \autocite{Tozzi2022} \\ \\
De z16 is ook `Quantum safe`: de bedreiging van Quantum computers in de toekomst blijft groeien en er is nog geen directe oplossing voor. IBM heeft hiervoor geïnvesteerd in Crypto Express 8S hardware security modules om data op de Mainframe te beschermen en Quantum safe te maken. De nieuwe modules bevatten nieuwe quantum safe encryptie algoritmes die geëvalueerd zijn door de US National Institute of Standards and Technology. \autocite{Sayer2022}
\\ \\
Dit systeem heeft natuurlijk een besturingssysteem en dit is z/OS. Hoewel er verschillende mogelijk zijn, is z/OS nog steeds het meest gebruikt. Dit is, samen met de IBM Z Systems, ontwikkeld door IBM waardoor dit het meest compatibel is met de hardware componenten en het blijft ondersteund door IBM zelf.

\subsection{Software in z/OS}
De software die wordt gebruikt is ook zeer verschillend. Hoewel er verschillende besturingssystemen mogelijk zijn, is z/OS nog steeds het meest gebruikt. Dit is, samen met de IBM Z Systems, ontwikkeld door IBM waardoor dit het meest compatibel is met de hardware componenten en het blijft ondersteund door IBM zelf. Dit bevat tools zoals Time Sharing Option (TSO), Interactive System Productivity Facility (ISPF) en System Display qnd Search Facility (SDSF) om er een paar op te noemen.

\subsection{Datasets}
z/OS heeft ook een andere manier om data op te slaan door middel van datasets. Dit is volgens de documentatie van \textcite{IBM} een collectie van gerelateerde data records dat opgeslagen en opgehaald wordt door een toegewezen naam. Dit zou u kunnen zien als een bestand in andere besturingssystemen zoals in Windows of Linux. \\ \\
Hier bestaan er verschillende soorten van maar de meest gebruikte is een sequentiële en partitionele dataset. \\ \\
Sequentiële datasets bevatten eigenlijk records die na elkaar opgeslagen zijn. Dit heeft als nadeel dat als je bijvoorbeeld record 20 wilt lezen, je eerst voorbij de voorgaande 19 records moet gaan. Dit is vooral nuttig om grote hoeveelheden data in op te slaan. \\ \\
Een partitionele dataset is meer voor programma's in te schrijven. Deze dataset bevat allemaal `members` die op hun beurt dan de effectieve data bevatten. Dit heeft als voordeel dat je in een pds de members kunt aanspreken in een willekeurige volgorde. Deze vorm van een dataset wordt ook wel een library genoemd.

\subsection{Unix op een mainframe}
IBM heeft veel ingezet op modernisatie. Zo is er een Unix Systems Services (USS) omgeving bijgekomen die naast de traditionele z/OS loopt. Dit is enkel een command line interface  Er is ook een volledige Mainframe die enkel Linux heeft als besturingssysteem namelijk de LinuxOne. In dit onderzoek zal er enkel gekeken worden naar een z16 die een USS omgeving heeft.

\section{IBM Developer for z/OS}
\label{sec:IBM Developer for z/OS (IDz)}
De testomgeving waarin we zullen werken is IBM Developer for z/OS (IDz) versie 16.0.2. Dit programma is volgens de definitie van \textcite{Spohn2023} een toolset voor het ontwikkelen en opzetten van hybride cloud applicaties op z/OS. \\
Het is een eclipse based programma met de mogelijkheid hebt om een connectie te maken met verschillende omgevingen op de mainframe (bv de USS of z/OS omgeving). Zo kunt u alle bestanden zien, openen en aanpassen. Omdat de data nogsteeds op de mainframe staat, kunnen wijzigingen direct gezien worden ookal bekijkt u het in een ander programma. Als u bijvoorbeeld een bestand wijzigt via IDz, kunt u de wijzigingen direct zien in ISPF. \\
Dit programma is vooral ontwikkelt om een envoudigere IDE te bieden om in te programmeren aangezien niet iedereen direct bekend is met ISPF. 