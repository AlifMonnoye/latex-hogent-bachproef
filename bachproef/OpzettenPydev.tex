%%=============================================================================
%% Opzetten Pydev in IDz
%%=============================================================================

\chapter{Opzetten Pydev}
\label{ch:opzetten-pydev}

Het programma waarin het onderzoek uitgevoerd zal worden is IBM Developer for z/OS (IDz) versie 16.0.2 met voorgeïnstalleerde programma's van DNB zelf. IDz is gebaseerd op Eclipse versie 4.23.0 . \\ In dit onderzoek zullen we versie 8.2.0 van PyDev installeren. Dit is een iets oudere versie maar het best compatibel met IDz, de andere versies kunnen problemen met zich meebrengen tijdens en na de installatie.
Versie 8.2.0 bevat onder andere een debugger, een code formatter en code analyse. Het is belangrijk om te weten dat IDz door IBM beschikbaar wordt gesteld en ze dus support zullen bezorgen waar nodig moesten er problemen zijn met hun programma. Dit is niet het geval met PyDev omdat dit door een externe partij is ontwikkeld en dus niet ondersteund wordt door IBM. \\

\section{Vereisten voor installatie} 
\\ \\
\begin{itemize}
    \item Eclipse versie 4.6 based programma nodig - IDz
    \item Java 11 of hoger
    \item Python 2.6 of hoger
\end{itemize}


\section{Installatie} 
\\ \\
Om Pydev te gebruiken moeten we het eerst installeren via hun Github repository \url{https://github.com/fabioz/Pydev/releases}. \\Dit zal een .zip installeren die je moet uitpakken en ergens moet opslaan. Het uitpakken kan een enige tijd duren. \\

In IDz navigeer je naar  \textit{help -> install New Software}  \\

Je zult een installatie wizard te zien krijgen en hierin klik je op \textit{add -> local} \\

Selecteer de map waar Pydev in is uitgepakt. klik op \textit{add} \\

Je krijgt weer een scherm te zien met de opties die je kunt installeren. Het kan zijn dat het eerst leeg is en dit verander je door in de checkbox de Group items by category uit te vinken. Hierna krijg je 3 opties.

\begin{itemize}
    \item PyDev for Eclipse 8.2.0.202102211157
    \item PyDev for Eclipse Developer Resources 8.2.0.202102211157
    \item PyDev Mylyn Integration 0.6.0
\end{itemize}

De eerste is noodzakelijk en de tweede optioneel. De derde mag niet aangevinkt worden aangezien MyLyn niet geïnstalleerd is in IDz waardoor het opzetten van PyDev zal falen. \\ 

Klik op \textit{next}, dit zal alles opzetten en kan even duren. \\

Als dit gedaan is wordt er een overzicht gegeven van de installatie details. Hier klik je op \textit{finish}. \\

In de balk rechtsonder wordt de status van de installatie weergegeven. \\

Als het klaar is met installeren zal het een pop-up geven om IDz herop te starten. Als u nog open projecten heeft is het best om deze eerst op te slaan en dan zelf te herstarten. Anders klikt u op \textit{Restart Now} \\ 

Om te zien of de installatie gelukt is navigeert u naar \textit{help -> about -> Installation details} \\

Geef in de zoekbalk \textquote{Pydev}. Hier zou u de geïnstalleerde software moeten zien \\ 

Als je een Python script opent zal het nogsteeds gebeuren in de interne text editor. Om dit te wijzigen moeten we de file associations bekijken. Deze zullen bepalen welke editor er gebruikt wordt bij een bepaalde extensie. Hier zullen we dus de Python editor van Pydev linken aan de .py extensie. \\

Ga naar \textit{Window -> Preferences -> General -> Editors -> File Associations} \\

Hier is een lijst met allemaal extensies en het programma waarmee ze geopend worden. IDz heeft niet standaard een python editor dus deze moet toegevoegd worden. Klik bij \textquote{file types} op \textit{add} \\

Geef hier .py in en klik op ok \\

In de lijst van extensies en ziet hier normaal .py tussenstaan \\

Selecteer de py extensie en bij \textquote{associated editors} komen er 3 opties:
\begin{itemize}
    \item Python Editor
    \item Text Editor
    \item Generic Text Editor
\end{itemize}
\\
Selecteer \textquote{Python Editor} en klik op \textit{Default}. \\


Als je een python script opent zal dit automatisch gebeuren in de python editor en zijn alle functies van Pydev beschikbaar.



