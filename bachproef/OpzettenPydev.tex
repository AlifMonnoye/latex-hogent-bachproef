%%=============================================================================
%% Opzetten Pydev in IDz
%%=============================================================================

\chapter{Opzetten Pydev}
\label{ch:opzetten-pydev}

In dit onderzoek zullen we een oudere versie vqn PyDev installeren namelijk PyDev 8.2.0 . Dit is een iets oudere versie maar het best compatibel met IDz, de andere versies kunnen problemen met zich meebrengen tijdens en na de installatie. Het programma waar het onderzoek in uitgevoerd zal worden is IBM Developer for z/OS (IDz) versie 16.0.2 met voorgeïnstalleerde programma's van DNB zelf. IDz is gebaseerd op Eclipse versie 4.23.0
Versie 8.2.0 bevat een debugger, een code formatter en code analyse. Het is belangrijk om te weten dat IDz door IBM beschikbaar wordt gesteld en ze dus support zullen bezorgen waar nodig moesten er problemen zijn met hun programma. Dit is niet het geval met PyDev omdat dit door een externe partij is ontwikkeld en dus niet gesupport is door IBM. \\

\section{prerequisites}
\begin{itemize}
    \item Eclipse (versie 4.6) based programma nodig - IDz
    \item Java 11 of hoger
    \item Python 3.8 of hoger
\end{itemize}

\\
\\
\section{Installatie}
Om Pydev te gebruiken moeten we het eerst installeren via hun Github repository \url{https://github.com/fabioz/Pydev/releases} . Dit zal een .zip installeren die je moet uitpakken en ergens moet opslaan. Het uitpakken kan een enige tijd duren. \\

In IDz nqvigeer je naar help -> install New Software . Je zult een installatie wizard te zien krijgen en hierin klik je op add -> local en hier kies je dan de map waar Pydev in is uitgepakt. klik op add.

Je krijgt weer een scherm te zien met de opties die je kunt installeren. Het kan zijn dat het eerst leeg is en dit verander je door vanonder in de checkbox de Group items by categorieuit te vinken. Hierna krijg je 3 opties.

PyDev for Eclipse 8.2.0.202102211157
PyDev for Eclipse Developer Resources 8.2.0.202102211157
PyDev Mylyn Integration 0.6.0

De eerste is noodzakelijk en de tweede optioneel. De derde mag niet aangevinkt worden aangezien MyLyn niet is geïnstalleerd in IDz waardoor het opzetten van PyDev zal falen.

Klik op next, dit zal alles opzetten en kan even duren. Als dit gedaan is krijg je een overzicht van de installatie details en hier klik je op finish. In de balk rechtsonder kun je zien dat het bezig is met installeren.

Als het klaar is met installeren zal het een pop-up geven om IDz herop te starten. Als u nog open projecten heeft is het best om deze eerst op te slaan en dan zelf te herstarten. Anders klikt u nu op Restart Now

Om te zien of de installatie gelukt is navigeert u naar help -> about -> Installation details en zoekt u Pydev.
Hier zou u de geïnstalleerde software moeten zien


Als je een Python script opent zal het nogsteeds gebeuren in de interne text editor. Om dit te wijzigen moeten we de file associations bekijken. Deze zullen bepalen welke editor er gebruikt wordt bij een bepaalde extensie. Hier zullen we dus de Python editor van Pydev linken aan de .py extensie.

Ga naar Windos -> Preferences -> General -> Editors -> File Associations
Hier zie je een lijst met allemaal extensies en het programma waarmee ze geopend worden. IDz heeft niet standaard een python editor dus deze moeten we toevoegen door op add te klikken bij file types
Geef hier .py in en klik op ok

Je komt terug in de lijst van extensies en ziet hier normaal .py tussenstaan

Selecteer de py extensie en bij associated editors komen er 3 opties.
Deze opties worden gebruikt om bestanden met een .py extensie te openen


Als je een python script opent zal dit automatisch gebeuren in de python editor en heb je alle functies die Pydev te bieden heeft.
%In IDz navigeer je naar help -> Eclipse Marketplace. Hierin vind je dus allemaal verschillende plug-ins die je kunt gebruiken in IDz. In de zoekbalk zoek je 'Pydev', klik je op enter en dan install. Bij confirm selected features krijg je 2 opties en deze moeten allebei aangeklikt zijn. Hierna klik je op confirm


