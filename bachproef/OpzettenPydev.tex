%%=============================================================================
%% Opzetten Pydev in IDz
%%=============================================================================

\chapter{Opzetten Pydev}
\label{ch:opzetten-pydev}

In dit onderzoek zullen we de nieuwste versie van Pydev installeren namelijk versie 12.0.0 .
Dit bevat nog functionaliteiten zoals sys.monitoring voor de debugger, ruff als een code formatter en betere code analyse. Het is belangrijk om te weten dat IDz door IBM beschikbaar wordt gesteld en ze dus support zullen bezorgen waar nodig moesten er problemen zijn met hun programma. Dit is niet het geval met PyDev omdat dit door een externe partij is ontwikkeld en dus niet gesupport is door IBM. \\
\section{prerequisites}
Voor de installatie van PyDev is een Eclipse (versie 4.6) based programma nodig. Het programma waar het onderzoek in uitgevoerd zal worden is IBM Developer for z/OS (IDz) versie 16.0.2 met voorgeïnstalleerde programma's van DNB zelf. IDz is gebaseerd op Eclipse versie 4.23.0. \\

Het is belangrijk dat Java 11 of hoger gebruikt wordt in IDz. Moest dit niet het geval zijn is er de mogelijkheid om oudere versies van Pydev te installeren. \\

Een Python versie 3.8 of hoger wordt enkel ondersteund en is nodig om Python programma's uit te voeren in IDz.
 
\\
\\
\section{Installatie}
Om Pydev te gebruiken moeten we het eerst installeren via hun Github repository \url{https://github.com/fabioz/Pydev/releases} . Dit zal een .zip installeren die je moet uitpakken en ergens moet opslaan. Het uitpakken kan een enige tijd duren. \\
In IDz navigeer je naar help -> Eclipse Marketplace. Hierin vind je dus allemaal verschillende plug-ins die je kunt gebruiken in IDz. In de zoekbalk zoek je 'Pydev', klik je op enter en dan install. Bij confirm selected features krijg je 2 opties en deze moeten allebei aangeklikt zijn. Hierna klik je op confirm


