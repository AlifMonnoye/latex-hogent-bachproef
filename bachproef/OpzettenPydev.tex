%%=============================================================================
%% Opzetten Pydev in IDz
%%=============================================================================

\chapter{Pydev opzetten}
\label{ch:opzetten-pydev}

Het programma waarin het onderzoek wordt uitgevoerd, is IBM Developer for z/OS (IDz) versie 16.0.2. IDz is gebaseerd op Eclipse versie 4.23.0 . \\ In dit onderzoek wordt versie 8.2.0 van PyDev geïnstalleerd. Dit is een oudere versie, maar ze is het best compatibel met IDz; de andere versies brengen problemen met zich mee tijdens en na de installatie.
Versie 8.2.0 bevat onder andere een debugger, een code formatter en code analyse. Het is belangrijk om te weten dat IDz door IBM beschikbaar wordt gesteld en dat IBM dus support zal bieden waar nodig mochten er problemen zijn met hun programma. Dit is niet het geval met PyDev omdat dit door een externe partij is ontwikkeld en dus niet ondersteund wordt door IBM. \\

\section{Vereisten voor installatie} 
\\ \\
\begin{itemize}
    \item Eclipse versie 4.6 based programma nodig - IDz;
    \item Java 11 of hoger;
    \item Python 2.6 of hoger.
\end{itemize}


\section{Installatie} 
\\ \\
Pydev kan geïnstalleerd worden via de Github repository \url{https://github.com/fabioz/Pydev/releases}. \\Hierdoor wordt een .zip geïnstalleerd die uitgepakt moet worden in een directory naar keuze. Het uitpakken kan enige tijd duren. \\

Navigeer in IDz naar \textit{help -> install New Software}  \\

Hier komt er een installatie wizard waar Pydev geïnstalleerd kan worden in IDz via \textit{add -> local}. \\

Selecteer de map waarin Pydev is uitgepakt. Klik vervolgens op \textit{add} \\

Er verschijnt een scherm met de installatie opties. Als dit leeg is, kan dit veranderd worden door de checkbox \textquote{Group items by category} uit te vinken. Hierna komen 3 opties tevoorschijn:

\begin{itemize}
    \item PyDev for Eclipse 8.2.0.202102211157;
    \item PyDev for Eclipse Developer Resources 8.2.0.202102211157;
    \item PyDev Mylyn Integration 0.6.0.
\end{itemize}

De eerste is noodzakelijk en de tweede is optioneel. De derde mag niet aangevinkt worden aangezien MyLyn niet geïnstalleerd is in IDz, waardoor het opzetten van PyDev zal mislukken. \\ 

Klik op \textit{next}, dit zal alles opzetten en kan even duren. \\

Wanneer dit gedaan is, wordt er een overzicht gegeven van de installatie details. Klik op \textit{finish} om de installatie te beginnen. \\

In de balk rechtsonder wordt de status van de installatie weergegeven. \\

Wanneer de installatie voltooid is, geeft IDz een pop-up om het programma opnieuw op te starten. Als er nog niet opgeslagen projecten open staan, worden deze best opgeslagen voor het vóór het heropstarten. Klik in het andere geval op \textit{Restart Now} \\ 

Navigeer om te zien of de installatie gelukt is naar \textit{help -> about -> Installation details}. \\

Geef in de zoekbalk \textquote{Pydev} in. Hier zou de geïnstalleerde software moeten staan. \\ 

Als er een Python script geopend wordt, zal dit nogsteeds gebeuren in de interne text editor. Om dit te wijzigen, moeten de file associations aangepast worden. Deze zullen bepalen welke editor er gebruikt wordt bij een bepaalde extensie. Hier wordt de Python editor van Pydev gelinkt aan de .py-extensie. \\

Ga naar \textit{Window -> Preferences -> General -> Editors -> File Associations} \\

Hier komt een lijst met allemaal extensies en het programma waarmee ze geopend worden. IDz heeft niet standaard een python-editor, dus deze moet toegevoegd worden. Klik bij \textquote{file types} op \textit{add}. \\

Geef hier \textquote{.py} in en klik op \textit{ok}. \\

In de lijst van extensies staat nu .py. \\

Selecteer de .py-extensie en bij \textquote{associated editors} komen er 3 opties:
\begin{itemize}
    \item Python Editor;
    \item Text Editor;
    \item Generic Text Editor.
\end{itemize}
\\
Selecteer \textquote{Python Editor} en klik op \textit{Default}. \\


Als een Python-applicatie geopend wordt, zal dit automatisch gebeuren in de Python-editor en zijn alle functies van Pydev beschikbaar.



