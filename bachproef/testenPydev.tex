%%=============================================================================
%% Testen runtime environment in IDz
%%=============================================================================
\chapter{Testen Runtime environment}
\label{ch:test-runtime}
In dit hoofdstuk zal er getest worden hoe krachtig de ingebouwde terminal van IDz is en wat zijn limitaties zijn. Dit gebeurd voor hoofdstuk \ref{ch:test-pydev} omdat in dit hoofdstuk de omgeving wordt opgegezet waarin de API uitgevoerd zal worden. \\
Er wordt verwacht dat Python correct geïnstalleerd is in de USS omgeving.

\section{Opzetten testomgeving}
Alle bestanden die worden geschreven om te testen zullen aangemaakt worden in de directory \textit{/tmp/BPtestfolder}. 
Deze wordt aangemaakt via de Bash commando's: 

\begin{lstlisting}
    $ mkdir /tmp/BPtestfolder
    $ cd /tmp/BPtestfolder
\end{lstlisting}

\section{Test script aanmaken en uitvoeren}
Om te testen of Python correct is geconfigureerd, is het best om een simpele applicatie te schrijven in deze programmeertaal. Door deze test wordt het ook duidelijk of er effectief applicaties uitgevoerd kunnen worden via IDz in USS. De test applicatie wordt als volgt aangemaakt:


\begin{lstlisting}
    $ touch test.py
    $ echo `print('Hello World')` >> test.py
    $ python test.py
    
\end{lstlisting}

In de pas aangemaakte directory wordt het bestand \textquote{test.py} aangemaakt wat \textquote{Hello World} zal afdrukken op het scherm. Aangezien dit geen groot programma is, zal dit met het \textit{echo} commando geschreven worden in het Python. Tijdens het uivoeren geeft dit programma \textquote{Hello World} terug en werkt dus perfect.

\section{Opzetten Python virtuele omgeving}
Een Python virtuele omgeving wordt vooral gebruikt om een eigen omgeving op te zetten waar de gebruiker toegang heeft tot alle functies zoals bijvoorbeeld packages installeren. Om de API uit te voeren zijn er verschillende packages nodig dus zal er gebruik worden gemaakt van zo een virtuele omgeving. Dit zal opgezet en geactiveerd worden in IDz. Als dit lukt zullen de nodige packages, die gebruikt worden in de test API, geïnstalleerd worden. \\

Om de virtuele omgeving aan te maken, wordt het python \textquote{venv} commando gebruikt. Deze omgeving wordt geactiveerd door middel van het \textquote{activate} script in de \textit{bin} subdirectory. Deze 2 stappen gebeuren als volgt:

\begin{lstlisting}
    $ python -m venv --system-site-packages ./virtualEnv
    $ cd virtualEnv/bin
    $ activate
    $ pip list
\end{lstlisting}

Hoewel de virtuele omgeving aangemaakt wordt, kan deze niet opgestart worden met het source commando wat wel gaat in een ssh connectie. In IDz moet er genavigeerd worden naar de \textit{bin} subdirectory en het \textit{activate} programma hierin uitvoeren. Om te werken met de virtuele omgeving moeten alle commando's in de \textit{bin} subdirectory gebeuren anders zal het niet lukken.
Het commando \textit{pip list} wordt gebruikt om te testen of de virtuele omgeving actief is. Moest het niet actief zijn, zou dit commando niet lukken. Het geeft een lijst van geïnstalleerde packages terug die geïnstalleerd zijn. Voorlopig zijn dit de packages die al op het systeem stonden. Bij het aanmaken van de virtuele omgeving werden deze overgenomen door middel van de \textit{--system-site-packages} optie.

\section{Installeren van packages}
Zoals eerder vermeld zijn er nog packages nodig waar Python programma's gebruik van kunnen maken. Deze zullen geïnstalleerd worden van de Python AI toolkit for z/OS of van de Pypi website. De packages die geïnstalleerd zullen worden zijn nodig voor het testprogramma dat wordt geschreven in hoofdstuk \ref{ch:test-pydev}. Deze packages zijn \textit{FastApi}, \textit{Uvicorn} en \textit{Jsonschema}. \\

FastAPI en Uvicorn maken geen onderdeel uit van de Python AI toolkit dus zullen geïnstalleerd moeten worden van \url{https://pypi.org/}. Het is belangrijk om alle packages waarvan FastAPI gebruik maakt, ook te installeren. Dit installatiebestand heeft als extensie \textquote{.whl} en via drag and drop is het mogelijk om deze in de USS omgeving te plaatsen vanaf de lokale machine. Dit moet in File Transfer Mode \textquote{binary} gebeuren omdat het anders niet correct geformateerd is. Dit kan ingesteld worden in IDz via \textit{Window -> Preferences -> Remote Systems -> Files}. Hier wordt een lijst van extensies weergegeven en hoe ze worden overgezet naar de USS omgeving. Om een .whl bestand correct over te zetten naar de mainframe, wordt \textit{Add...} geselecteerd. Hier is het mogelijk om een extensie van een bestand in te geven en in dit geval is dit \textquote{.whl}. Dit bestandstype komt in de lijst terecht en als dit geselecteerd is, moet de \textit{Default File Transfer Mode} op \textit{Binary} staan. Nu is het mogelijk om installatiebestanden in USS te krijgen in het juiste formaat. \\

Jsonschema is beschikbaar via de Python AI toolkit en staat dus al op de mainframe. Dit is ook een .whl bestand. \\

Er zal ook gebruik worden gemaakt van de ZOAU maar deze zijn al geïnstalleerd op het systeem en hebben geen verdere configuratie nodig. \\

Alle nodige packages zullen opgeslagen worden in de directory \textit{/tmp/BPtestfolder/virtualE nv/packages}. Via het \textit{cp} commando worden de packages uit de Python AI toolkit gekopiëerd naar de aangemaakte directory: \\

\begin{lstlisting}
    $ mkdir /tmp/BPtestfolder/virtualEnv/packages
    $ cp /Path/To/AI/Toolkit/jsonschema-4.17.3-py3-none-any.whl \
                                         /tmp/BPtestfolder/virtualEnv/packages
\end{lstlisting} 

Al deze packages moeten apart geïnstalleerd worden via het \textit{pip install} commando. Aangezien deze bestanden op het systeem staan, wordt de optie \textit{--no-index} toegevoegd. Dit laat het systeem weten dat het niet online moet zoeken om de packages te installeren. De installatie gebeurd als volgt:\\
\begin{lstlisting}
    $ pip install ../packages/annotated_types-0.4.0-py3-none-any.whl --no-index
    $ pip install ../packages/typing_extensions-4.10.0-py3-none-any.whl --no-index
    $ pip install ../packages/pydantic-1.10.15-py3-none-any.whl --no-index
    $ pip install ../packages/idna-3.6-py3-none-any.whl --no-index
    $ pip install ../packages/sniffio-1.3.1-py3-none-any.whl --no-index
    $ pip install ../packages/anyio-4.3.0-py3-none-any.whl --no-index
    $ pip install ../packages/starlette-0.35.1-py3-none-any.whl --no-index
    $ pip install ../packages/fastapi-0.109.0-py3-none-any.whl --no-index
    
    $ pip install ../packages/h11-0.14.0-py3-none-any.whl --no-index
    $ pip install ../packages/uvicorn-0.25.0-py3-none-any.whl --no-index
\end{lstlisting}

\begin{lstlisting}
    $ pip install ../packages/attrs-23.2.0-py3-none-any.whl --no-index
    $ pip install ../packages/pyrsistent-0.20.0-py3-none-any.whl --no-index
    $ pip install ../packages/jsonschema-4.17.3-py3-none-any.whl --no-index
    
\end{lstlisting}

\section{Omgevingsvariabelen instellen}
Zoals eerder vermeld, zal er gebruik worden gemaakt van ZOAU om datasets te lezen en schrijven. Deze package is al geïnstalleerd maar er zullen nog omgevingsvariabelen ingesteld moeten worden zodat dit correct werkt. De omgeveringsvariabelen die gewijzigd moeten worden zijn \textit{ZOAU\_HOME}, \textit{PATH} en \textit{LIBPATH}. Hiervoor gebruiken we het \textit{export} commando als volgt: \\

\begin{lstlisting}
    $ export ZOAU_HOME=/pp/idz/v16r0/zoautil/
    $ export PATH=$ZOAU_HOME/bin:$PATH
    $ export LIBPATH=$ZOAU\_HOME/lib:$LIBPATH
    
\end{lstlisting}

Moesten deze variabelen niet gewijzigd zijn, zou het systeem volgende foutmelding geven:
\begin{lstlisting}
    CEE3201S The system detected an operation exception 
                                        (System Completion Code=0C1).
    From entry point _zoau_io_zopen at compile unit offset +0000000029B242B4 
                 at entry offset +00000000000002D4 at address 0000000029B242B4.
    Killed
\end{lstlisting}

%%=============================================================================
%% Testen runtime environment in IDz
%%=============================================================================
\chapter{Testen Pydev}
\label{ch:test-pydev}

Om Pydev te testen, zal er een API in IDz geschreven worden. Deze API heeft als functie om van een batch job, de JCL Job card en JCL Procedure aan elkaar te schrijven in een PDS member. Hierdoor wordt het gebruik van geïnstalleerde packages alsook de verbinding met z/OS getest. \\

Het doel van dit hoofdstuk blijft de ervaring testen om een volledig Python programma te schrijven door middel van Pydev en zien hoe efficiënt dit is. Er zal gebruik worden gemaakt van de ZOAU om datasets te lezen en schrijven. Er is ook een kleine JSON file aanwezig die de naam van de job card -en procedure JCL bevat. \\

Door deze verschillende technologieën te gebruiken, wordt niet enkel getest hoe complexe programma's er geschreven kunnen worden in Pydev, maar ook hoe efficiënt het is om op de mainframe gebruik te maken van deze moderne technieken in combinatie met de oudere.

\section{Schrijven van een test API}
Er werd een test API geschreven. Deze bevat een functie die 2 JCL's uit het JSON bestand van een meegegeven job ophaald en aan elkaar schrijft. Dit zal geschreven worden in een dataset die gedefinieerd is in de variabele \textquote{dsMemberPath}. Deze functie maakt gebruik van 2 andere functies om het JSON bestand te lezen en om te schrijven naar een dataset. De API werd geschreven als volgt: \\

\begin{lstlisting}
    from fastapi import FastAPI
    import json
    from zoautil_py import zsystem, datasets
    from zoautil_py.zoau_io import zopen
    
    app = FastAPI()
    
    @app.get("/api/proc/{proc}")
    def get_jcl(proc: str):
        # ======================================================== #
        # ================= Declaring variables ================== #
        # ======================================================== #
        
        # header_path = path of the jobcard
        # body_path = path of the procedure
        header_path, body_path = get_values()
        
        # List to store the lines in of both the job card and procedure
        records = []
        
        # Output dataset
        outds = "AD12215.API.OUT" # Dataset
        member = proc # Dataset member
        dsMemberPath = f"{outds}({member})" # Full path of the dataset member
        
        
        # ======================================================== #
        # ============ Check if input datasets exist ============= #
        # ======================================================== #
        # Both files exist
        if (proc in datasets.list_members(header_path)) \
                            & (proc in datasets.list_members(body_path)):
            print(f"Both members found ({member}), \
                returning both the job card and procedure")
        
            # Reading and appending JOB file
            with zopen(f"//'{header_path}({proc})'", 'r', "cp1047") \
                            as record_stream:
                for record in record_stream:
                    records.append(record.upper())
        
            # Reading and appending PROC file
            with zopen(f"//'{body_path}({proc})'", 'r', "cp1047") \
                            as record_stream:
                for record in record_stream:
                    records.append(record.upper())
        
        # Only header file exists
        elif (proc in datasets.list_members(header_path)) \
                            & (proc not in datasets.list_members(body_path)):
            print(f"Member {body_path}({proc}) not found in procedure path, \
                            returning only {header_path}({proc})")
            
            # Reading and appending JOB file
            with zopen(f"//'{header_path}({proc})'", 'r', "cp1047") \
                            as record_stream:
                for record in record_stream:
                    records.append(record.upper())
        
        # Only body file exists
        elif (proc not in datasets.list_members(header_path)) \
                            & (proc in datasets.list_members(body_path)):
            print(f"Member {header_path}({proc}) not found in job card, \
                            returning only {body_path}({proc})")
            
            # Reading and appending PROC file
            with zopen(f"//'{body_path}({proc})'", 'r', "cp1047") \
                            as record_stream:
                for record in record_stream:
                    records.append(record.upper())
                    
        # No files exist 
        else:
            return "files don't exist"
        
        # ======================================================== #
        # ============ Writing output into a dataset ============= #
        # ======================================================== #
        
        # Deleting older output written if it exists
        if member in datasets.list_members(outds):
            datasets.delete_members(f"{dsMemberPath}")
            if member in datasets.list_members(outds):
                return f"Dataset member {dsMemberPath} \
                            can't be deleted because it is open"
                
        # Writing the content of "records" into the dataset "dsMemberPath"   
        return write_into_dataset(dsMemberPath, records)
    
    
    def get_values():
    
        # Opening the json file with the file paths 
        with open("config.json", 'r', encoding="utf-8") as test_file:
        
            # Loading the json in the variable json_data
            json_data = json.load(test_file)
            
            # Assign the return values with the correct filepaths
            header_path = json_data["HEADER"]
            body_path = json_data["BODY"] 
            return header_path, body_path
    
    
    def write_into_dataset(dsname: str, records: list):
        for record in records:
            # Remove all control characters for the record
            mapping = dict.fromkeys(range(32))
            res = record.translate(mapping)
            # Writing record into dataset
            datasets.write(dsname, res, append=True)
            
        return f"Output written in PDS {dsname}"
    
    
\end{lstlisting}
\vspace{5 mm}
Bij het oproepen van de API geeft de gebruiker de naam van de job mee. Dit wordt opgeslagen in de variabele \textquote{proc}. Van deze job wordt de job card JCL en procedure JCL opgehaald door middel van de functie \textquote{get\_values}. Hier wordt een JSON bestand gelezen die de naam van de PDS bevat waar de JCL members in opgeslagen zijn. Dit is een andere dataset voor de job card en de procedure. Het JSON bestand ziet er als volgt uit: \\

\begin{lstlisting}
    {
        {
            `HEADER`: ``
            `BODY`: ``
        }
    }
\end{lstlisting}

Hierna moet er gecontroleerd worden of deze JCL's bestaan. Er bestaat een \textquote{datasets.exists} functie in ZOAU maar dit is vooral bedoeld om te controleren of een sequentiële dataset bestaat. Omdat de JCL bestanden in een PDS member wordt opgeslagen, kan dit niet gebruikt worden. Hierdoor wordt er gebruik gemaakt van de \textquote{datasets.list\_members} functie. Zoals de naam vermeld, geeft dit een lijst van members in een PDS terug. Als de naam van de job in deze lijst zit, bestaat de JCL. \\
 
Om de datasets te lezen, wordt er gebruik gemaakt van het \textit{zopen} statement uit de ZOAU. Dit is het equivalent van het traditionele \textit{open} statement in Python om een bestand te openen. De \textquote{zopen} functie opent dus een dataset en hier worden er 3 positionele parameters gebruikt: 
\begin{itemize}
    \item[1] Het te openen bestand. Dit bevat de naam van de dataset.
    \item[2] Dit bepaald wat er gebeurd wordt met het bestand. \textit{\textquote{r}} voor te lezen en \textit{\textquote{w}} voor te schrijven
    \item[3] De encoding van het meegegeven bestand
\end{itemize} 
Het is belangrijk om de encoding mee te geven bij het oproepen van deze functie. Datasets worden opgeslagen in de encoding \textit{IBM-1047} en dit wordt gedefinieerd als \textit{'cp1047'} \\

Om de data te schrijven in een PDS member, zullen alle records in volgorde opgeslagen worden in een list. De job card en procedure zitten in dezelfde list genaamd \textquote{records}. Eerst wordt de job card opgehaald en daarna de procedure. Deze records worden dan geschreven in de PDS member gedefinieerd in de variabele \textquote{dsMemberPath}. Voor het schrijven, wordt er gebruik gemaakt van de functie \textquote{write\_into\_dataset} die 2 argumenten meekrijgt: de naam van de PDS member waarin er geschreven wordt en de lijst met alle records. Deze functie maakt gebruik van de ZOAU functie \textquote{datasets.write}. Als deze dataset niet bestaat, wordt deze automatisch aangemaakt.


\section{Uitvoeren van de volledige applicatie}
Om de API uit te voeren zal er gebruik worden gemaakt van de Uvicorn package met het volgend Bash commando:

\begin{lstlisting}
    $ python -m uvicorn main:app
    
\end{lstlisting}
Dit commando moet uitgevoerd worden in dezelfde directory als het python bestand. De API zal hierdoor luisteren op poort 8000 van het loopback adres van z/OS voor inkomende verzoeken. \\

Om te testen of het Python programma werkt, wordt er een nieuwe Python applicatie ontwikkeld die uitgevoerd wordt terwijl de API aan het luisteren is voor binnenkomende verzoeken. Voor deze applicatie wordt de \textquote{requests} package gebruikt. 

\begin{lstlisting}
    import requests
    
    url = `http://127.0.0.1/api/proc/bn379`
    
    x = requests.get(url)
    print(x)
\end{lstlisting}

In deze applicatie wordt er een request verzonden naar het loopback adres van de mainframe met de correcte url die de job \textquote{bn379} bevat. 

\chapter{Evaluatie test}
\label{ch:eval-test}
Deze test applicatie maakte gebruik van de meeste functies waarvoor Python bedoeld is in combinatie met de mainframe. Een API wordt vaak geschreven in Python en kan dus ook uitgevoerd worden op de mainframe. Het lezen van JSON bestanden is ook een zeer belangrijke functie. Dit in combinatie met het gebruik van datasets op z/OS maakt Python alleen maar krachtiger. \\

Het schrijven van de API ging vrij vlot met Pydev waardoor het zeer nuttig is om deze plug-in te implementeren. Dit werkte perfect met code completion voor traditionele statements zoals \textquote{return} of \textquote{with}. Als er variabelen gedeclareerd werden, konden deze ook automatisch ingevuld worden. Het werkt zoals een IDE zou moeten werken. \\

Hoewel het schrijven van de API vlot ging, was het uitvoeren in IDz niet zeer efficiënt. Er waren problemen met het opstarten van de Python virtuele omgeving wat niet voorkwam in een externe ssh connectie. De oorzaak van dit probleem is nog niet helemaal duidelijk. Het uitvoeren van de API bracht ook problemen met zich mee. Dit maakt het duidelijk dat het testen van Python applicaties in de IDz terminal lukt als deze niet zeer complex zijn.







