%%=============================================================================
%% Runtime environment opzetten
%%=============================================================================

\chapter{Runtime environment opzetten}
\label{ch:runtime-environment}
\section{Python-applicaties uitvoeren op een lokale machine via IDz}
In IDz is er de mogelijkheid om programma's uit te voeren op de lokale machine waarop IDz uitgevoerd wordt. Hiervoor moet Python geïnstalleerd zijn op deze machine en in IDz moet ernaar worden verwezen. \\

DNB heeft strenge regels voor het installeren van externe software op laptops die een connectie kunnen maken met de mainframe. Door deze veiligheidsredenen is er geen mogelijkheid om Python te installeren en kan deze stap ook niet uitgevoerd worden. \\

Als dit mogelijk is, kan er naar de Python executable verwezen worden in IDz via \textit{Window -> preferences -> Pydev -> Interpreters -> Python Interpreter}. \\
Door te klikken op \textit{New...}, verschijnt er een wizard om de interpreter te kiezen. Zoek de geïnstalleerde Python-interpreter en klik op \textit{Apply}. \\

Door rechts te klikken op een Python-bestand en naar \textit{Run} te navigeren, kan het programma uitgevoerd worden. \\
Vermoedelijk zal dit het Python-applicatie uitvoeren op de gebruikte machine en niet in de USS-omgeving op de mainframe. 


\section{Python-applicaties uitvoeren in de USS omgeving via IDz}
De Python-interpreter is niet nodig op de lokale machine voor het uitvoeren van Python programma's in de USS-omgeving. Hiervoor moet het remote systems explorer perspectief geopend worden via \textit{Window -> Perspective -> Open Perspective -> Other -> Remote System Explorer}. \\

Maak een connectie met de USS-omgeving met de juiste inloggegevens en navigeer naar het tabblad \textit{Remote Shell}. \\

Dit zal leeg zijn, maar door te klikken op de 3 puntjes met als naam \textit{View Menu} kan de connectie toegevoegd worden. Klik op \textit{Launch} en kies de juiste connectie. Dit zou 2 connecties moeten tonen. De eerste is de connectie met de lokale machine en de tweede is die met de USS-omgeving. Selecteer de connectie met de USS omgeving, dit zal 2 keuzes geven:

\begin{itemize}
    \item[1] z/OS UNIX Shells;
    \item[2] TSO Commands.
\end{itemize}

Om connectie te maken met de USS-omgeving, wordt optie 1 gekozen: \textit{z/OS UNIX Shells}. De tweede optie is om TSO-commando's te geven in de z/OS-omgeving waarmee een connectie is gemaakt. \\

Hiermee wordt een terminal opgestart waar er commando's ingegeven kunnen worden in de USS-omgeving. Hierdoor kan er genavigeerd worden naar verschillende directories met Python bestanden. Deze kunnen worden uitgevoerd met het Python-commando. Dit heeft ook als functie om directories of bestanden aan te maken in deze terminal zonder een externe ssh-connectie nodig is. \\

%Belangrijk om te vermelden is dat de Python installer moet zijn gedefinieerd anders kun je geen Python bestanden uitvoeren in USS. Dit is een globale variabele die je moet initializeren.