%%=============================================================================
%% Conclusie
%%=============================================================================

\chapter{Conclusie}%
\label{ch:conclusie}

% TODO: Trek een duidelijke conclusie, in de vorm van een antwoord op de
% onderzoeksvra(a)g(en). Wat was jouw bijdrage aan het onderzoeksdomein en
% hoe biedt dit meerwaarde aan het vakgebied/doelgroep? 
% Reflecteer kritisch over het resultaat. In Engelse teksten wordt deze sectie
% ``Discussion'' genoemd. Had je deze uitkomst verwacht? Zijn er zaken die nog
% niet duidelijk zijn?
% Heeft het onderzoek geleid tot nieuwe vragen die uitnodigen tot verder 
%onderzoek?

Deze proof of concept biedt een meerwaarde aan de modernisatie van het mainframe systeem die er volgens mij zit aan te komen. Deze systemen zijn nog heel belangrijk maar werkt met te oude technieken waardoor er niet veel mensen zijn die ze kunnen onderhouden. Het toevoegen van modernere manieren van werken is niet voldoende, ze moeten ook efficiënt zijn om mee te kunnen werken wat niet het geval is in de standaard versie van IDz. \\

Het opstellen van deze plug-in is niet zeer complex maar er zijn een paar instellingen die niet voor de hand liggend zijn en voor problemen kunnen zorgen. Aangezien IDz ook veel verschillende functionaliteiten heeft weten veel mensen niet waar ze moeten zoeken als ze iets zoals dit willen opzetten. \\

Het is zeer makkelijk om alles in 1 IDE te hebben tijdens het programmeren omdat je niet telkens van programma naar programma moet springen. Persoonlijk vind ik de terminal in Idz om de Python scripts in uit te voeren niet veel beter om in te werken en verkies zelf liever een ssh connectie via powershell. 
Tijdens het testen heb ik gemerkt dat iets complexere zaken niet direct lukken zoals een Python virtuele omgeving opzetten. In de USS omgeving maakt dit het moeilijk om packages te installeren die gebruikt worden in bijna alle Python programma's. \\

Zelf had ik niet verwacht dat het mogelijk was om Python programma's uit te voeren in USS via IDz en was deels verbaasd toen ik ondervond dat het wel kon. Dit toont voor mij nog eens aan hoe groot Eclipse based programma's zoals IDz kunnen zijn. 

