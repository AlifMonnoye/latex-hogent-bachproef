%%=============================================================================
%% Voorwoord
%%=============================================================================

\chapter*{\IfLanguageName{dutch}{Woord vooraf}{Preface}}%
\label{ch:voorwoord}

%% TODO:
%% Het voorwoord is het enige deel van de bachelorproef waar je vanuit je
%% eigen standpunt (``ik-vorm'') mag schrijven. Je kan hier bv. motiveren
%% waarom jij het onderwerp wil bespreken.
%% Vergeet ook niet te bedanken wie je geholpen/gesteund/... heeft

Dit is een proof of concept die ik wil uitwerken omdat dit hulp kan bieden bij de modernisatie van de mainframe systemen van IBM. Op het moment van schrijven heb ik al stage gedaan bij 2 bedrijven die gebruikmaken van een mainframe; zij gebruiken allebei IBM Developer for zOS om applicaties te schrijven in verschillende soorten programmeertalen zoals COBOL, PL/1, Python en Java. Python is momenteel één van de meest gebruikte programmeertalen en IBM doet veel inspanning om dit zo efficiënt mogelijk te laten werken op zijn systemen. De skills in de `oudere` programmeertalen zoals COBOL en PL/1 zijn sterk aan het afnemen dus is de inbreng van Python bijna een must om nog goede programmeurs te vinden. Een goede editor is ook nodig wat momenteel niet het geval is in IDz. In dit onderzoek maak ik dan ook een stappenplan om dit zo goed mogelijk op te stellen.
\\ \\
Ik zou Njaal Sletten en Stian Botnevik van DNB, Noorwegen willen bedanken voor de hulp en inspiratie bij het schrijven en uitwerken van dit onderzoek. Ze hielpen allebei met het technische gedeelte en Stian kwam met de probleemstelling.