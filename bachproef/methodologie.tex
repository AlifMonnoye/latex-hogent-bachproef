%%=============================================================================
%% Methodologie
%%=============================================================================

\chapter{Methodologie}%
\label{ch:methodologie}

%% TODO: In dit hoofstuk geef je een korte toelichting over hoe je te werk bent
%% gegaan. Verdeel je onderzoek in grote fasen, en licht in elke fase toe wat
%% de doelstelling was, welke deliverables daar uit gekomen zijn, en welke
%% onderzoeksmethoden je daarbij toegepast hebt. Verantwoord waarom je
%% op deze manier te werk gegaan bent.
%% 
%% Voorbeelden van zulke fasen zijn: literatuurstudie, opstellen van een
%% requirements-analyse, opstellen long-list (bij vergelijkende studie),
%% selectie van geschikte tools (bij vergelijkende studie, "short-list"),
%% opzetten testopstelling/PoC, uitvoeren testen en verzamelen
%% van resultaten, analyse van resultaten, ...
%%
%% !!!!! LET OP !!!!!
%%
%% Het is uitdrukkelijk NIET de bedoeling dat je het grootste deel van de corpus
%% van je bachelorproef in dit hoofstuk verwerkt! Dit hoofdstuk is eerder een
%% kort overzicht van je plan van aanpak.
%%
%% Maak voor elke fase (behalve het literatuuronderzoek) een NIEUW HOOFDSTUK aan
%% en geef het een gepaste titel.

\section{Fase 1: Literatuurstudie}
\label{sec:m-literatuurstudie}
\begin{itemize}
    \item \textbf{Doel:}
    Relevante informatie verzamelen van IDz, Pydev en de Unix System Services omgeving.
    \item \textbf{Aanpak:}
    \item[-] Opzoeken betrouwbare bronnen
    \item[-] Gelijke projecten onderzoeken
    \item[-] Overzicht van nodige software
    \item \textbf{Tijd:} 5 weken
    \item \textbf{Opbrengst:}
    Een volledig onderzoek naar vakliteratuur en nodige software die nodig is. Ook een volledige schets naar de omgeving waarin er wordt gewerkt.
\end{itemize}


\section{Fase 2: Opstellen Pydev}
\label{sec:m-opstellen-pydev}
\begin{itemize}
    \item \textbf{Doel:}
    Opzetten van de Pydev plugin in IDz
    \item \textbf{Aanpak:}
    \item[-] De opgezochte literatuur bestuderen met expert
    \item[-] Pydev opstellen met de juiste vereisten
    \item \textbf{Tijd:} 3 weken
    \item \textbf{Opbrengst:}
    Pydev voor functies zoals code completion, syntax check, code analysis, ... \\ 
    Bij het openen van een Python programma zal dit ook gebeuren via een Python editor
\end{itemize}


\section{Fase 3: Runtime environment opzetten}
\label{sec:m-python-interpreter-opzetten}
\begin{itemize}
    \item \textbf{Doel:}
    Connectie maken met de USS omgeving in IDz om Python programma's in uit te voeren
    \item \textbf{Aanpak:}
    \item[-] De opgezochte literatuur bestuderen met expert
    \item[-] Connectie met de USS omgeving opzetten
    \item[-] Testen
    
    \item \textbf{Tijd:} 2 weken
    \item \textbf{Opbrengst:}
    Python programma's kunnen uitvoeren in USS via IDz.
\end{itemize}


\section{Fase 4: Analyse functies van Pydev}
\label{sec:m-verdere-configuratie}
\begin{itemize}
    \item \textbf{Doel:}
    Onderzoeken welke functies er beschikbaar zijn in Pydev
    \item \textbf{Aanpak:}
    \item[-] Pydev documentatie bekijken
    \item[-] Zelf onderzoeken in IDz
    
    \item \textbf{Tijd:} 2 weken
    \item \textbf{Opbrengst:}
    Een overzicht van alle functies die beschikbaar zijn. 
\end{itemize}

\section{Fase 5: Evaluatie voorbereiding}
\label{sec:m-evaluatie-voorbereiding}
\begin{itemize}
    \item \textbf{Doel:}
    Eind evaluatie voorbereiden
    \item \textbf{Aanpak:}
    \item[-] Resultaat bespreken, vergelijken en concluderen
    \item[-] Op orde stellen van nodige documenten
    \item \textbf{Tijd:} 2 weken
    \item \textbf{Opbrengst:}
    Een eind evaluatie waarin de opdracht besproken wordt
\end{itemize}


