%%=============================================================================
%% Methodologie
%%=============================================================================

\chapter{Methodologie}%
\label{ch:methodologie}

%% TODO: In dit hoofstuk geef je een korte toelichting over hoe je te werk bent
%% gegaan. Verdeel je onderzoek in grote fasen, en licht in elke fase toe wat
%% de doelstelling was, welke deliverables daar uit gekomen zijn, en welke
%% onderzoeksmethoden je daarbij toegepast hebt. Verantwoord waarom je
%% op deze manier te werk gegaan bent.
%% 
%% Voorbeelden van zulke fasen zijn: literatuurstudie, opstellen van een
%% requirements-analyse, opstellen long-list (bij vergelijkende studie),
%% selectie van geschikte tools (bij vergelijkende studie, "short-list"),
%% opzetten testopstelling/PoC, uitvoeren testen en verzamelen
%% van resultaten, analyse van resultaten, ...
%%
%% !!!!! LET OP !!!!!
%%
%% Het is uitdrukkelijk NIET de bedoeling dat je het grootste deel van de corpus
%% van je bachelorproef in dit hoofstuk verwerkt! Dit hoofdstuk is eerder een
%% kort overzicht van je plan van aanpak.
%%
%% Maak voor elke fase (behalve het literatuuronderzoek) een NIEUW HOOFDSTUK aan
%% en geef het een gepaste titel.

\section{Fase 1: Literatuurstudie}
\label{sec:m-literatuurstudie}
In deze fase zal er relevante informatie verzameld worden over relevante vakliteratuur. Dit zal gedaan worden door betrouwbare bronnen op te zoeken en te bestuderen. \\
De tijd die hiervoor gereserveerd wordt, is 5 weken en zal als resultaat een volledig overzicht geven van de omgeving waarin dit onderzoek zich bevindt. 

%\begin{itemize}
%    \item \textbf{Doel:}
%    Relevante informatie verzamelen van IDz, Pydev en de Unix System Services omgeving.
%    \item \textbf{Aanpak:}
%    \item[-] Opzoeken betrouwbare bronnen
%    \item[-] Gelijke projecten onderzoeken
%    \item[-] Overzicht van nodige software
%    \item \textbf{Tijd:} 5 weken
%    \item \textbf{Opbrengst:}
%    Een volledig onderzoek naar vakliteratuur en nodige software die nodig is. Ook een volledige schets %naar de omgeving waarin er wordt gewerkt.
%\end{itemize}


\section{Fase 2: Opstellen Pydev}
\label{sec:m-opstellen-pydev}
Het doel in deze fase is het implementeren van Pydev in IDz door de opgezochte literatuur te bestuderen en te onderzoeken. Eens dit gedaan is, zal de implementatie van Pydev beginnen over een periode van 3 weken. Na deze implementatie zullen Python applicaties geopend kunnen worden met Pydev voor functies zoals code completion, syntax check en code analyse. 
%\begin{itemize}
%    \item \textbf{Doel:}
%    Opzetten van de Pydev plugin in IDz
%    \item \textbf{Aanpak:}
%    \item[-] De opgezochte literatuur bestuderen met expert
%    \item[-] Pydev opstellen met de juiste vereisten
%    \item \textbf{Tijd:} 3 weken
%    \item \textbf{Opbrengst:}
%    Pydev voor functies zoals code completion, syntax check, code analysis, ... \\ 
%    Bij het openen van een Python programma zal dit ook gebeuren via een Python editor
%\end{itemize}


\section{Fase 3: Runtime environment opzetten}
\label{sec:m-python-interpreter-opzetten}
In deze fase zal er een connectie gemaakt worden met de USS omgeving zodat de Python applicaties daar uitgevoerd kunnen worden via IDz. Dit zal 2 weken duren en gebruikers kunnen dan via IDz hun Python applicaties uitvoeren in de USS omgeving.
%\begin{itemize}
%    \item \textbf{Doel:}
%    Connectie maken met de USS omgeving in IDz om Python programma's in uit te voeren
%    \item \textbf{Aanpak:}
%    \item[-] De opgezochte literatuur bestuderen met expert
%    \item[-] Connectie met de USS omgeving opzetten
%    \item[-] Testen
%    
%    \item \textbf{Tijd:} 2 weken
%    \item \textbf{Opbrengst:}
%    Python programma's kunnen uitvoeren in USS via IDz.
%\end{itemize}


\section{Fase 4: Testen van Pydev en connectie met USS}
\label{sec:m-verdere-configuratie}
In de testfase zal er een testomgeving opgezet worden in USS via IDz om zo de mogelijkheden en limitaties van deze connectie te achterhalen. In IDz wordt er een Python API ontwikkeld om te testen hoe efficiënt Pydev is voor het programmeren in Python. Hiervoor zijn 2 weken vrijgehouden en zal als resultaat de voor -en nadelen terug geven. 
%\begin{itemize}
%    \item \textbf{Doel:}
%    Onderzoeken welke functies er beschikbaar zijn in Pydev
%    \item \textbf{Aanpak:}
%    \item[-] Pydev documentatie bekijken
%    \item[-] Zelf onderzoeken in IDz
%    
%    \item \textbf{Tijd:} 2 weken
%    \item \textbf{Opbrengst:}
%    Een overzicht van alle functies die beschikbaar zijn. 
%\end{itemize}

\section{Fase 5: Evaluatie voorbereiding}
\label{sec:m-evaluatie-voorbereiding}
In deze fase valt de conclusie van heel de bachelorproef en het voorbereiden op de eindevaluatie. Dit duurt 2 weken.
%\begin{itemize}
%    \item \textbf{Doel:}
%    Eind evaluatie voorbereiden
%    \item \textbf{Aanpak:}
%    \item[-] Resultaat bespreken, vergelijken en concluderen
%    \item[-] Op orde stellen van nodige documenten
%    \item \textbf{Tijd:} 2 weken
%    \item \textbf{Opbrengst:}
%    Een eind evaluatie waarin de opdracht besproken wordt
%\end{itemize}


