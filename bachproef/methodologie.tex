%%=============================================================================
%% Methodologie
%%=============================================================================

\chapter{Methodologie}%
\label{ch:methodologie}

%% TODO: In dit hoofstuk geef je een korte toelichting over hoe je te werk bent
%% gegaan. Verdeel je onderzoek in grote fasen, en licht in elke fase toe wat
%% de doelstelling was, welke deliverables daar uit gekomen zijn, en welke
%% onderzoeksmethoden je daarbij toegepast hebt. Verantwoord waarom je
%% op deze manier te werk gegaan bent.
%% 
%% Voorbeelden van zulke fasen zijn: literatuurstudie, opstellen van een
%% requirements-analyse, opstellen long-list (bij vergelijkende studie),
%% selectie van geschikte tools (bij vergelijkende studie, "short-list"),
%% opzetten testopstelling/PoC, uitvoeren testen en verzamelen
%% van resultaten, analyse van resultaten, ...
%%
%% !!!!! LET OP !!!!!
%%
%% Het is uitdrukkelijk NIET de bedoeling dat je het grootste deel van de corpus
%% van je bachelorproef in dit hoofstuk verwerkt! Dit hoofdstuk is eerder een
%% kort overzicht van je plan van aanpak.
%%
%% Maak voor elke fase (behalve het literatuuronderzoek) een NIEUW HOOFDSTUK aan
%% en geef het een gepaste titel.

\subsection{Fase 1: Literatuurstudie}
\begin{itemize}
    \item \textbf{Doel:}
    Relevante informatie verzamelen van IDz, Pydev en de Unix System Services omgeving.
    \item \textbf{Aanpak:}
    \item[-] Opzoeken betrouwbare bronnen
    \item[-] Gelijke projecten onderzoeken
    \item[-] Overzicht van nodige software
    \item[-] Opmaken stappenplan voor installatie Pydev
    \item \textbf{Tijd:} 4 weken
    \item \textbf{Opbrengst:}
    Een volledig onderzoek naar vakliteratuur en nodige software die nodig is. Ook een volledige schets naar de omgeving waarin er wordt gewerkt.
\end{itemize}


\subsection{Fase 2: Opstellen Pydev}
\begin{itemize}
    \item \textbf{Doel:}
    Opzetten van de Pydev plugin in IDz
    \item \textbf{Aanpak:}
    \item[-] De opgezochte literatuur bestuderen
    \item[-] Pydev opstellen met de juiste vereisten
    \item[-] Testen
    \item \textbf{Tijd:} 2 weken
    \item \textbf{Opbrengst:}
    Pydev voor functies zoals code completion, syntax check, code analysis, ... \\ 
    Bij het openen van een Python programma zal dit ook gebeuren via een Python editor
\end{itemize}


\subsection{Fase 3: Python interpreter opzetten}
\begin{itemize}
    \item \textbf{Doel:}
    Python Interpreter opstellen om Python programma's uit te voeren in IDz.
    \item \textbf{Aanpak:}
    \item[-] De opgezochte literatuur bestuderen
    \item[-] Opzetten van de juiste interpreter
    \item[-] Configuratie om output van de uitgevoerde code in de IDz console te kunnen zien
    \item[-] Testen
    
    \item \textbf{Tijd:} 2 weken
    \item \textbf{Opbrengst:}
    Python programma's kunnen uitvoeren in IDz.
\end{itemize}


\subsection{Fase 4: Verdere configuratie om de output van de mainframe in de IDz console te krijgen}
\begin{itemize}
    \item \textbf{Doel:}
    Zorgen dat de output van de scripts weergegeven wordt in de console van IDz
    \item \textbf{Aanpak:}
    \item[-] De opgezochte literatuur bestuderen
    \item[-] Verdere configuraties toepassen
    
    \item \textbf{Tijd:} 4 week
    \item \textbf{Opbrengst:}
    Output in IDz om sneller code te testen. 
\end{itemize}

\subsection{Fase 5: Evaluatie voorbereiding}
\begin{itemize}
    \item \textbf{Doel:}
    Een volledige vergelijking van de opgenomen resultaten.
    \item \textbf{Aanpak:}
    \item[-] Resultaat bespreken
    \item[-] Op orde stellen van nodige documenten
    \item \textbf{Tijd:} 2 weken
    \item \textbf{Opbrengst:}
    Een volledig overzicht en conclusie van de gerealiseerde opdracht met bijhorende opmerkingen en analyses. 
\end{itemize}


