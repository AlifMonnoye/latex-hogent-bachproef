\chapter{\IfLanguageName{dutch}{Stand van zaken}{State of the art}}%
\label{ch:stand-van-zaken}

% Tip: Begin elk hoofdstuk met een paragraaf inleiding die beschrijft hoe
% dit hoofdstuk past binnen het geheel van de bachelorproef. Geef in het
% bijzonder aan wat de link is met het vorige en volgende hoofdstuk.

% Pas na deze inleidende paragraaf komt de eerste sectiehoofding.

\section{Een andere manier van werken}
\label{sec:andere-manier-van-werken}
Veel bedrijven die werken met een mainframe hebben het probleem dat ze onvoldoende mensen vinden met de nodige kennis over deze technologie. De introductie van Unix systemen op de mainframe in het jaar 2000 \autocite{Mertic2020} had dit probleem wat verminderd maar zeker niet weggewerkt. De oorzaak van deze situatie is door de oude technieken die het systeem gebruikt. COBOL en PL/1 zijn niet de meest aantrekkelijke programmeertalen om te leren en mensen kiezen liever Bash als scriptingtaal in plaats van REXX. Hoewel IBM veel inzet op documentatie en online leerplatformen zoals IBM Z Xplore, blijft het tekort van ervaren mensen nog steeds te laag. \\

De mainframe wereld zal zich dus moeten aanpassen aan de vaardigheden van de mensen door over te schakelen naar beter gekende manieren van werken. 
Python bijvoorbeeld is een programmeertaal die steeds meer populariteit krijgt sinds zijn ontstaan in de vroege jaren 90. Dit is niet enkel bij reeds ervaren programmeurs, maar mensen die net beginnen programmeren kiezen hier steeds vaker voor. Dit is vooral door zijn begin vriendelijkheid, verschillende doeleinden en een actieve community. \autocite{Johnson2023}
IBM heeft dit opgemerkt en een Python- compiler en interpreter ontwikkeld voor z/OS genaamd IBM Open Enterprise SDK for Python. Hierdoor kun je met Python interactie hebben met z/OS om bijvoorbeeld applicaties te ontwikkelen of de resources van het systeem beheren. Door de Unix omgeving heeft de programmeur ook geen geavanceerde z/OS kennis nodig. \autocite{Klaey2023}

\section{Bedrijven in het werkveld}
\label{sec:bedrijven-in-werkveld}
Een bank is een goed voorbeeld van een bedrijf die gebruik maakt van IBM hun Z Systems. Dit valt te concluderen door het feit dat maar liefst 92 van de top 100 banken wereldwijd een mainframe gebruikt. Dit is ook logisch aangezien er gemiddeld 12.6 miljard financiële transacties per dag zijn dat door deze systemen worden uitgevoerd. \autocite{Wagle2017} \\
Door dit aantal is de nood voor een mainframe toch niet te onderdrukken maar het is niet enkel het aantal transacties dat deze machine kan uitvoeren, het is ook de snelheid, schaalbaarheid en beveiliging van deze systemen dat een grote rol speelt. \\
Het zijn niet enkel banken die van deze technologie gebruik maken, maar ook  verzekeringsmaatschappijen bijvoorbeeld. De top 10 van deze soort bedrijven maken allemaal gebruik van een IBM Mainframe \autocite{Tozzi2022}

\section{De Skill gap}
\label{sec:skill-gap}
Het is nog steeds moeilijk om nieuw talent aan te trekken in de mainframe wereld. Een onderzoek van  \textcite{Deloitte2020} 
toont aan dat 79\% van de projectleiders moeilijkheden heeft met het zoeken naar mensen met de juiste skillset. Hetzelfde onderzoek toont ook aan dat er in de teams zelf een groot verschil is van kennis en vaardigheden.  \\ Hoewel dit systeem gebruikt wordt door 71\% van de fortune 500 companies \autocite{Tozzi2022} , is de Skill gap nog steeds te groot waardoor veel bedrijven vrezen voor een groot tekort aan werknemers om deze Z Systems te onderhouden.
\\
Volgens Petra Goude zijn er verschillende manieren om dit probleem aan te pakken. Zo kunnen bedrijven lessen geven over Mainframe en hoe ze dit gebruiken. Ze vertelt ook dat de vaardigheden die nodig zijn beter gecomplimenteerd moeten worden en kunnen leiden naar een belonende en lange termijn carrière. \\ Hoewel dit zou helpen, vind ze dit niet de kern van het probleem. Het zijn de oude technieken die mensen niet aantrekt. Ze stelt voor om meer te investeren in hedendaagse technologie en dit te installeren op de mainframe. Dit kan gaan over dezelfde test -en deployment technieken, maar ook over hedendaagse programmeertalen zoals Java of Python. Dit zou kunnen door middel van APIs en zou een nieuwe, jongere werkkracht aantrekken. \autocite{Goude2023}


\section{Toch nog een kleurrijke toekomst}
\label{sec:kleurrijke-toekomst}
Ondanks deze probleemstelling, ziet de toekomst er toch nog goed uit voor deze systemen. Zo wordt er meer gemoderniseerd met bijvoorbeeld modernere programmeertalen. Er wordt ook voorspeld dat we een introductie van DevOps en self-service benaderingen gaan zien. \autocite{Pennaz2023} \\
Momenteel zijn Python en Java beschikbaar als programmeertaal op de mainframe naast PL/1 en COBOL.
Sinds deze introductie zijn al bijna 2/3de van gebruikers op de mainframe Java aan het toepassen op een bepaalde manier. \autocite{Watts2018} 


%Je verwijst bij elke bewering die je doet, vakterm die je introduceert, enz.\ naar je bronnen. In \LaTeX{} kan dat met het commando \texttt{$\backslash${textcite\{\}}} of \texttt{$\backslash${autocite\{\}}}. Als argument van het commando geef je de ``sleutel'' van een ``record'' in een bibliografische databank in het Bib\LaTeX{}-formaat (een tekstbestand). Als je expliciet naar de auteur verwijst in de zin ( narratieve referentie), gebruik je \texttt{$\backslash${}textcite\{\}}. Soms is de auteursnaam niet expliciet een onderdeel van de zin, dan gebruik je \texttt{$\backslash${}autocite\{\}} (referentie tussen haakjes). Dit gebruik je bv.~bij een citaat, of om in het bijschrift van een overgenomen afbeelding, broncode, tabel, enz. te verwijzen naar de bron. In de volgende paragraaf een voorbeeld van elk.

%\textcite{Knuth1998} schreef een van de standaardwerken over sorteer- en zoekalgoritmen. Experten zijn het erover eens dat cloud computing een interessante opportuniteit vormen, zowel voor gebruikers als voor dienstverleners op vlak van informatietechnologie~\autocite{Creeger2009}.

%Let er ook op: het \texttt{cite}-commando voor de punt, dus binnen de zin. Je verwijst meteen naar een bron in de eerste zin die erop gebaseerd is, dus niet pas op het einde van een paragraaf.

