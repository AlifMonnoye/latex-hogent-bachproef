%%=============================================================================
%% Samenvatting
%%=============================================================================

% TODO: De "abstract" of samenvatting is een kernachtige (~ 1 blz. voor een
% thesis) synthese van het document.
%
% Een goede abstract biedt een kernachtig antwoord op volgende vragen:
%
% 1. Waarover gaat de bachelorproef?
% 2. Waarom heb je er over geschreven?
% 3. Hoe heb je het onderzoek uitgevoerd?
% 4. Wat waren de resultaten? Wat blijkt uit je onderzoek?
% 5. Wat betekenen je resultaten? Wat is de relevantie voor het werkveld?
%
% Daarom bestaat een abstract uit volgende componenten:
%
% - inleiding + kaderen thema
% - probleemstelling
% - (centrale) onderzoeksvraag
% - onderzoeksdoelstelling
% - methodologie
% - resultaten (beperk tot de belangrijkste, relevant voor de onderzoeksvraag)
% - conclusies, aanbevelingen, beperkingen
%
% LET OP! Een samenvatting is GEEN voorwoord!

%%---------- Nederlandse samenvatting -----------------------------------------
%
% TODO: Als je je bachelorproef in het Engels schrijft, moet je eerst een
% Nederlandse samenvatting invoegen. Haal daarvoor onderstaande code uit
% commentaar.
% Wie zijn bachelorproef in het Nederlands schrijft, kan dit negeren, de inhoud
% wordt niet in het document ingevoegd.

%\IfLanguageName{english}{%
%\selectlanguage{dutch}
%\chapter*{Samenvatting}
%\lipsum[1-4]
%\selectlanguage{english}
%}{}

%%---------- Samenvatting -----------------------------------------------------
% De samenvatting in de hoofdtaal van het document

\chapter*{Samenvatting}
IBM heeft voor zijn IBM Z Systems de deur opengezet naar vernieuwing door Java en Python toegankelijk te maken op de Mainframe. Hoewel deze talen ondersteund worden, is het niet altijd even makkelijk om programma's hierin te schrijven. IBM Developer for z/OS is een stap in de goede richting door zijn functionaliteit om in een bekend programma bestanden te openen en aan te passen die opgeslagen zijn op de mainframe. Aangezien dit Eclipse based is, is er geen Python interpreter waardoor Python scripts aangepast moeten worden in een text editor. \\

Deze paper zal een proof of concept zijn voor het opstellen van Pydev en een Python IDE in IBM Developer for z/OS. Hierdoor kunnen Python developers efficiënter code schrijven en aanpassen in de Unix System Services omgeving van de mainframe. \\

In dit onderzoek zal er gewerkt worden met de Pydev plug-in voor Eclipse en zullen er instructies gegeven worden over hoe dit ingesteld wordt. Ook zal er onderzocht worden of deze scripts uitgevoerd kunnen worden via IBM Developer for z/OS in de mainframe omgeving. \\

Om dit onderzoek te testen, zal er een Python API geschreven worden in IBM Developer for z/OS. Deze zal uitgevoerd worden via dit programma op de Unix System Services omgeving. Deze omgeving brengt andere problemen met zich mee die opgelost zullen worden. \\

Het resultaat zal een volledige Proof of concept zijn voor een Python interpreter op te zetten in dit programma. Verder volgt configuratie om een verbinding te maken met de Unix System Services omgeving om Python applicaties in uit te voeren. In de testfase van dit onderzoek, wordt een API geschreven en een stappenplan gegeven over wat er nodig is om Python applicaties uit te voeren in de gebruikte omgeving. Dit is belangrijk voor de programmeurs van DNB die werken in dit programma en voorlopig hun Python code moeten testen via een externe ssh connectie.
