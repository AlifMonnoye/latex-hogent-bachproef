%==============================================================================
% Sjabloon onderzoeksvoorstel bachproef
%==============================================================================
% Gebaseerd op document class `hogent-article'
% zie <https://github.com/HoGentTIN/latex-hogent-article>

% Voor een voorstel in het Engels: voeg de documentclass-optie [english] toe.
% Let op: kan enkel na toestemming van de bachelorproefcoördinator!
\documentclass{hogent-article}

% Invoegen bibliografiebestand
\addbibresource{voorstel.bib}

% Informatie over de opleiding, het vak en soort opdracht
\studyprogramme{Professionele bachelor toegepaste informatica}
\course{Bachelorproef}
\assignmenttype{Onderzoeksvoorstel}
% Voor een voorstel in het Engels, haal de volgende 3 regels uit commentaar
% \studyprogramme{Bachelor of applied information technology}
% \course{Bachelor thesis}
% \assignmenttype{Research proposal}

\academicyear{2023 - 2024} 

\title{PoC: Opzetten van Pydev in IBM Developer for z/OS om Python programma's uit te voeren op de mainframe}

\author{Alif Monnoye}
\email{alif.monnoye@student.hogent.be}

% TODO: Geef de co-promotor op
\supervisor[Co-promotor]{ Njaal Sletten, \email{njaal.sverre.slette@dnb.no} } 
%J. Doe (X, \href{mailto:sigrid.beekman@synalco.be}{sigrid.beekman@synalco.be})}


\specialisation{Mainframe Expert}
\keywords{Python, Mainframe, Pydev, IDz, USS}

\begin{document}

\begin{abstract}
    IBM heeft voor zijn IBM Z Systems de deur opengezet naar vernieuwing door Java en Python beschikbaar te maken op de mainframe. Hoewel deze programmeertalen ondersteund worden, is het niet altijd even efficiënt om programma's hierin te schrijven. IBM Developer for z/OS is een stap in de goede richting door zijn functionaliteit om in een bekend programma bestanden te openen en aan te passen die opgeslagen zijn op de mainframe. Aangezien dit Eclipse based is, heeft dit programma geen Python interpreter waardoor Python scripts aangepast moeten worden in een ingebouwde text editor. \\
    
    Deze paper zal een proof of concept zijn voor het opstellen van Pydev en een Python interpreter in IBM Developer for z/OS. Hierdoor kunnen Python developers efficiënter code schrijven, aanpassen en uitvoeren in de Unix System Services omgeving van de mainframe. Verder zal er nog onderzocht worden om de output van de Python programma's te zien in de console van IBM Developer for z/OS. Dit is belangrijk voor mainframe programmeurs van Den Norske Bank (DNB) die werken in dit programma en momenteel hun Python code moeten testen in een ssh connectie. \\
    Er zal gewerkt worden met de Pydev plug-in voor Eclipse en er zullen stap voor stap instructies gegeven worden over hoe dit ingesteld zal zijn. Ook zal er onderzocht worden of deze scripts uitgevoerd kunnen worden in IBM Developer for z/OS door middel van een connectie met de mainframe omgeving. \\
    
    Het resultaat zal een volledige Proof of concept zijn voor een Python interpreter op te zetten in dit programma. Dit is vooral relevant voor Python developers in bedrijven die IBM Developer for z/OS gebruiken in combinatie met een IBM z Mainframe. 
\end{abstract}

\tableofcontents

% De hoofdtekst van het voorstel zit in een apart bestand, zodat het makkelijk
% kan opgenomen worden in de bijlagen van de bachelorproef zelf.
%---------- Inleiding ---------------------------------------------------------

\section{Introductie}%
\label{sec:introductie}
In de huidige wereld van technologie is het moeilijk bij te houden welke mogelijkheden er zijn om bepaalde projecten uit te voeren. Zo heb je altijd nieuwe technieken die net iets efficiënter of krachtiger zijn dan een ander. Door deze voortgaande evolutie zullen individuen niet direct op de hoogte zijn van recente ontwikkelingen, maar vergeten ze ook de recente toepassingen van een bepaalde techniek. Dit is vooral merkbaar bij de IBM Z Systems die zo belangrijk, maar zo snel vergeten worden in de IT wereld. Hoewel het een zeer hoogwaardige technologie is, zijn de technieken nog steeds zeer oud. Hierdoor is het moeilijk om mensen te vinden met de juiste kwalificaties maar dit kan mogelijks weggewerkt worden door DevOps teams een weg te geven naar programmeren op de mainframe.


\subsection{Probleemstelling}
In deze vergelijkende studie zal er gekeken worden naar 2 verschillende CI/CD pipelines die gebruikt kunnen worden in een mainframe omgeving: een Gitlab -en Jenkins pipeline. 
Gitlab is een web-based Git repository en heeft een groot bereik van open, gratis en privé repositories. Dit wordt vooral gebruikt voor het automatiseren van taken in een software project op een DevOps platform. Het heeft verschillende extra functies zoals managen, monitoren en beveiliging de bron code. \autocite{Mohanan2023} \\
Jenkins langs de andere kant is een open-source automatisatie tool voor CI/CD. Dit is één van de meest gebruikte DevOps tools door zijn snelheid en kracht om software te bouwen, testen en inzetten. \autocite{Sharma2023} \\

CI/CD is bedoeld voor het bouwen, testen en inzetten van software applicaties. Deze tool is vooral handig om in real-time te weten of je code werkt. Hierdoor zullen DevOps teams weinig tijd verliezen aan slechte code en constante fix requests terwijl ze ondertussen al aan een ander project bezig zijn. \autocite{Poberezhnyk2023} \\
Deze pipelines dienen als toegangspoort tussen het DevOps team door een API en het besturingssysteem van de mainframe. De resultaten van de scripts worden opgeslagen op een DB2 databank die al geconfigureerd is op het systeem.
 
Het bedrijf waar ik op ga focussen is Den Norske Bank (DNB) met een hoofdvestiging in Oslo, Noorwegen. Ze zoeken een manier om met beter gekende technologie te werken zoals Python. Door de Python compiler van IBM zal het uitvoeren van Python scripts geen probleem zijn. \\
Het is dan belangrijk om te weten wat de sterktes en zwaktes zijn van de verschillende pipelines zodat DNB hier een goed overzicht van heeft. Zo kunnen ze dan zelf bepalen wat ze het beste toepassen in hun systeem. Er zou mogelijks ook met een combinatie van de 2 gewerkt worden moesten Gitlab en Jenkins zich verschillend gedragen op verschillende type projecten. \\

Het resultaat van de vergelijking zal zich baseren op de volgende punten:
%
%\begin{itemize}
%    \item Sterktes van beide pipelines
%    \item Zwaktes van elke pipeline
%    \item Gebruiksvriendelijkheid
%    \item Beschikbare documentatie
%    \item Totale kost
%\end{itemize}

%---------- Stand van zaken ---------------------------------------------------

\section{State-of-the-art}%
\label{sec:state-of-the-art}

\subsection{Een andere manier van werken}
Veel bedrijven die werken met een mainframe hebben het probleem dat ze onvoldoende mensen vinden met de nodige kennis over deze technologie. De introductie van Unix systemen op de mainframe in het jaar 2000 \autocite{Mertic2020} had dit probleem wat verminderd maar zeker niet weggewerkt. De oorzaak van deze situatie is door de oude technieken die het systeem gebruikt. COBOL en PL/1 zijn niet de meest voor de hand liggende programmeertalen om te leren en mensen kiezen liever Bash als scriptingtaal in plaats van REXX. Hoewel IBM veel inzet op documentatie en online leerplatformen zoals IBM Z Xplore, blijft het tekort van ervaren mensen nog steeds te laag. \\

De mainframe wereld zal zich dus moeten aanpassen aan de vaardigheden van de mensen door over te schakelen naar bekendere manieren van werken. 
Python bijvoorbeeld is een programmeertaal die steeds meer populariteit krijgt sinds zijn creatie in de vroege jaren 90. Dit is niet enkel bij reeds ervaren programmeurs, maar mensen die net beginnen programmeren kiezen hier steeds vaker voor. Dit is vooral door zijn begin vriendelijkheid, verschillende doeleinden en een actieve community. \autocite{Johnson2023}
IBM heeft dit opgemerkt en een Python- compiler en interpreter ontwikkeld voor z/OS genaamd IBM Open Enterprise SDK for Python. Hierdoor kun je met Python interactie hebben met z/OS om bijvoorbeeld applicaties te ontwikkelen of de resources van het systeem gaan beheren. De programmeur heeft ook geen speciale z/OS kennis nodig, als er toegang is tot een moderne DevOps omgeving. \autocite{Klaey2023}

\subsection{Bedrijven in het werkveld}
Een bank is een goed voorbeeld van een bedrijf die gebruik maakt van IBM hun Mainframes. Dit valt te concluderen door het feit dat maar liefst 92 van de top 100 banken wereldwijd een mainframe gebruikt. Dit is ook logisch aangezien er gemiddeld 12.6 miljard financiële transacties per dag zijn. \autocite{Wagle2017} \\
Door dit aantal is de nood voor een mainframe toch niet te onderdrukken maar het is niet enkel het aantal transacties dat deze machine kan uitvoeren, het is ook de snelheid, schaalbaarheid en beveiliging dat een grote rol speelt. \\
Het zijn niet enkel banken die van deze technologie gebruik maken, maar ook  verzekeringsmaatschappijen bijvoorbeeld. De top 10 van deze soort bedrijven maken allemaal gebruik van een IBM Mainframe \autocite{Tozzi2022}

\subsection{De Skill gap}
Het is nog steeds moeilijk om nieuw talent aan te trekken in de mainframe wereld. Een onderzoek van \textcite{Deloitte2020} toont aan dat 79\% van de projectleiders moeilijkheden heeft met het zoeken naar mensen met de juiste skillset. Hetzelfde onderzoek toont ook aan dat er in de teams zelf een groot verschil is van kennis en vaardigheden.  \\ Hoewel dit systeem gebruikt wordt door 71\% van de fortune 500 companies \autocite{Tozzi2022} , is de Skill gap nog steeds te groot waardoor veel bedrijven vrezen voor een groot tekort aan werknemers om deze Z Systems te onderhouden.
\\
Volgens Petra Goude zijn er verschillende manieren om dit probleem aan te pakken. Zo kunnen bedrijven lessen geven over Mainframe en hoe ze dit gebruiken. Ze vertelt ook dat de vaardigheden die nodig zijn beter gecomplimenteerd moeten worden en kunnen leiden naar een belonende en lange termijn carrière. \\ Hoewel dit zou helpen, vind ze dit niet de kern van het probleem. Het zijn de oude technieken die mensen niet aantrekt. Ze stelt voor om meer te investeren in hedendaagse technologie en dit te installeren op de mainframe. Dit kan gaan over dezelfde test -en deployment technieken, maar ook over hedendaagse programmeertalen zoals Java of Python. Dit zou kunnen door middel van APIs en zou een nieuwe, jongere werkkracht aantrekken. \autocite{Goude2023}


\subsection{Toch nog een kleurrijke toekomst}
Hoewel er een grote nood is aan mensen met de juiste vaardigheden voor Mainframe, ziet de toekomst er toch nog goed uit. Zo wordt er meer gemoderniseerd met bijvoorbeeld modernere programmeertalen. Er wordt ook voorspeld dat we een introductie van DevOps en self-service benaderingen gaan zien. \autocite{Pennaz2023} \\
Momenteel is Java beschikbaar als programmeertaal op de mainframe naast PL/1 en COBOL.
Sinds deze introductie tot nu zijn al bijna 2/3de van gebruikers op de mainframe Java aan het toepassen op een bepaalde manier. \autocite{Watts2018}


% Voor literatuurverwijzingen zijn er twee belangrijke commando's:
% \autocite{KEY} => (Auteur, jaartal) Gebruik dit als de naam van de auteur
%   geen onderdeel is van de zin.
% \textcite{KEY} => Auteur (jaartal)  Gebruik dit als de auteursnaam wel een
%   functie heeft in de zin (bv. ``Uit onderzoek door Doll & Hill (1954) bleek
%   ...'')


%---------- Methodologie ------------------------------------------------------
\section{Methodologie}%
\label{sec:methodologie}
Deze paper zal als resultaat een volledige vergelijken geven over een Gitlab -en Jenkins pipeline om Python scripts uit te voeren op een mainframe in een bankomgeving. 

\subsection{Fase 1: Literatuurstudie}
\begin{itemize}
    \item \textbf{Doel:}
          Volledige werking van een CI/CD tools op de mainframe onderzoeken en bespreken.
    \item \textbf{Aanpak:}
          \item[-] Opzoeken betrouwbare bronnen
          \item[-] Gelijke projecten onderzoeken
          \item[-] Opstellen vereisten testomgeving
          \item[-] Opmaken stappenplan voor installatie pipelines
          
     \item \textbf{Tijd:} 4 weken
     \item \textbf{Opbrengst:}
           Een volledig onderzoek naar vakliteratuur en een samenvatting van gelijkaardige projecten en hun manier van werken. Op basis van de onderzochte projecten zal er ook een stappenplan gemaakt worden voor de pipelines op te zetten.  
\end{itemize}


\subsection{Fase 2: Opstellen testomgeving}
\begin{itemize}
    \item \textbf{Doel:}
    Een bruikbare test omgeving opstellen in het Mainframe systeem om de Gitlab -en Jenkins pipeline op te zetten.
    \item \textbf{Aanpak:}
    \item[-] De opgezochte literatuur bestuderen
    \item[-] Een testomgeving opstellen met de juiste vereisten
    \item \textbf{Tijd:} 2 weken
    \item \textbf{Opbrengst:}
    Een volledige testomgeving met de juiste vereisten om de pipelines in op te zetten.
\end{itemize}


\subsection{Fase 3: Pipelines + API opzetten}
\begin{itemize}
    \item \textbf{Doel:}
    Een Gitlab -en Jenkins pipeline opstellen waar we onze Python scripts op kunnen zetten om op de Mainframe uit te voeren. Verder ontwikkel ik nog een bruikbare API voor de gebruikers.
    \item \textbf{Aanpak:}
    \item[-] De opgezochte literatuur bestuderen
    \item[-] Ontwerpen pipelines + API
    \item[-] Implementeren pipelines
    \item[-] Implementeren API 
    
    \item \textbf{Tijd:} 5 weken
    \item \textbf{Opbrengst:}
    Volledige implementatie van Gitlab -en Jenkins pipeline en bijhorende API.
\end{itemize}


\subsection{Fase 4: Bruikbaarheid van de pipelines testen}
\begin{itemize}
    \item \textbf{Doel:}
    Zorgen dat de output van de scripts worden opgeslagen in de DB2 databank
    \item \textbf{Aanpak:}
    \item[-] Scripts runnen en output bekijken
    \item[-] Wijzigingen uitvoeren waar nodig
    
    \item \textbf{Tijd:} 1 week
    \item \textbf{Opbrengst:}
    Een connectie tussen de pipelines en de mainframe voor een correcte uitvoering van de scripts. 
\end{itemize}

\subsection{Fase 5: Evaluatie voorbereiding}
\begin{itemize}
    \item \textbf{Doel:}
    Een volledige vergelijking van de opgenomen resultaten.
    \item \textbf{Aanpak:}
    \item[-] Resultaat bespreken
    \item[-] Op orde stellen van nodige documenten
    \item \textbf{Tijd:} 2 weken
    \item \textbf{Opbrengst:}
    Een volledig overzicht en conclusie van de gerealiseerde opdracht met bijhorende opmerkingen en analyses. 
\end{itemize}


%---------- Verwachte resultaten ----------------------------------------------
\section{Verwacht resultaat, conclusie}%
\label{sec:verwachte_resultaten}
Een volledig overzicht van de praktische verschillen tussen de Gitlab -en Jenkins pipeline. Dit zal verdeeld worden in verschillende onderdelen:

\begin{itemize}
    \item Sterktes van beide pipelines
    \item Zwaktes van elke pipeline
    \item Gebruiksvriendelijkheid
    \item Beschikbare documentatie
    \item Totale kost
\end{itemize}

Hierbij zullen we bespreken waar elke pipeline in uitblinkt en eerder wat stilvalt. Dit doen we zodat we kunnen concluderen voor welk soort projecten elk product het best voor gebruikt wordt. Dit maakt het makkelijker voor de software engineers bij bedrijven zoals DNB om te bepalen welk soort pipeline ze kunnen gebruiken op basis van de taken die er vooral uitgevoerd zullen worden door de pipeline. 


\part{Literatuurstudie}
\section{CI/CD Pipeline}
Eerst zullen we moeten bepalen wat een CI/CD pipeline is. CI staat voor Continuous Integration wat het maken en testen van de software automatiseert. Dit is verdeeld in een bouw -en teststage. In de bouwstage zullen development teams hun code ontwikkelen op hun lokale computer. Als je het naar de repository wilt brengen kan dit natuurlijk wat fouten met zich meebrengen vooral door de verschillende omgeving waar de code in is gemaakt met de verschillende tools die gebruikt worden. Het bouwproces bevat wel tools om de software kwaliteit en omgeving te automatiseren en onderhouden. \autocite{Fosco2022} \\

De test stage is volledig geautomatiseerd door de pipeline omdat dit een redelijk repetitief proces is. Hierdoor slaan developers dit vaak over waardoor er niet kwalitatieve code naar de gedeelde repository gestuurd kan worden. De automatisatie van de pipeline vermijdt dit. \autocite{Fosco2022} \\

CD kan voor 2 dingen staan, het eerste is Continuous delivery. Dit staat voor de verantwoordelijk van de pipeline om code te plaatsen in een gedeelde repository. Het is dus van belang dat het integratie deel zo correct en zorgvuldig mogelijk wordt uitgevoerd. 

CD kan ook voor Continuous Deployment staan. Dit is nog een stap verder dan delivery omdat het automatisch de wijzigingen op productie kan zetten waardoor het bruikbaar is voor de klant \autocite{RedHat2023} 

\section{Gitlab Pipeline}
Aangezien DNB al een Gitlab pipeline heeft opgezet in hun DevOps omgeving, zullen we dit vooral gaan analyseren.

\section{Jenkins Pipeline}
Jenkins is een open source CI/CD pipeline die developers kunnen integreren in hun development omgeving. Het is geschreven in de programmeertaal Java en is beschikbaar over verschillende besturingssystemen zoals Linux en Windows in een GUI of CLI. Door de verschillende mogelijkheden is deze tool zeer populair om te gebruiken door DevOps teams. Niet alleen de flexibiliteit van systemen speelt een grote rol, maar ook het feit dat het volledig open-source is met een community van developers. Hierdoor zijn er veel verschillende plugins die je kunt gebruiken met voldoende documentatie waardoor iedereen er makkelijk gebruik van kan maken. \autocite{Khan2021} \\
De steun komt niet enkel van developers die eraan werken in hun vrije tijd, maar ook door donaties van grote technologie bedrijven zoals Github, AWS, CloudBees, IBM en nog veel meer. Het is niet enkel geld dat deze bedrijven te bieden hebben, maar ook hun eigen services. Github steunt bijvoorbeeld meer dan 2600 repositories wat meer dan een miljoen lijnen code bevat. \autocite{CdFoundation2023} \\

Deze tool was oorspronkelijk deel van een project genaamd Hudson door Oracle. Hier wordt niet meer aan gewerkt. Jenkins wordt gezien als een open source project onder begeleiding van de CD Foundation, wat valt onder de Linux Foundation. \autocite{Riglian2019} \\
Vandaag de dag zijn er meer dan 1800 plugins beschikbaar en in 2022 heeft Jenkins 300.000 gekende installaties bereikt wat het de meest opgezette automation server maakt. Verder is het aantal maandelijkse Jenkins pipeline jobs met 79\% gestegen van juni 2022 tot juni 2023.  \autocite{CdFoundation2023}


\part{Opstellen testomgeving}
\section{Jenkins Pipeline}
Dit is de configuratie van de Jenkins Pipeline


\part{Pipelines + API opzetten}
\section{Jenkins API}
Configuratie van de Jenkins pipeline en API

\part{Bruikbaarheid van de pipelines testen}
\section{Vergelijking}
Hier vergelijken we de pipelines


\part{Evaluatie voorbereiding}
\section{Volledige vergelijking}
Volledig overzicht waar Gitlab en Jenkins in verschillen    
\newpage
\printbibliography[heading=bibintoc]

\cleardoublepage
%\usepackage{graphicx}
\begin{figure}[pt!]
    \centering
    \includegraphics[width=550pt]{GanttChart_Updated.png}
    \caption{Gantt Chart}
    \label{fig}
\end{figure}


\end{document}