%==============================================================================
% Sjabloon poster bachproef
%==============================================================================
% Gebaseerd op document class `a0poster' door Gerlinde Kettl en Matthias Weiser
% Aangepast voor gebruik aan HOGENT door Jens Buysse en Bert Van Vreckem

\documentclass[a0,portrait]{hogent-poster}

% Info over de opleiding
\course{Bachelorproef}
\studyprogramme{toegepaste informatica}
\academicyear{2023-2024}
\institution{Hogeschool Gent, Valentin Vaerwyckweg 1, 9000 Gent}

% Info over de bachelorproef
\title{Implementeren van Pydev in IBM Developer for z/OS om Python-applicaties uit te voeren op de mainframe.}
% \subtitle{Ondertitel (eventueel)}
\author{Alif Monnoye}
\email{alif.monnoye@student.hogent.be}
\supervisor{Leendert Blondeel}
\cosupervisor{Njaal Sletten (DNB)}

% Indien ingevuld, wordt deze informatie toegevoegd aan het einde van de
% abstract. Zet in commentaar als je dit niet wilt.
\specialisation{Mainframe Expert}
\keywords{Python, Mainframe, Pydev, IDz, USS}
\projectrepo{https://github.com/AlifMonnoye/latex-hogent-bachproef}

\begin{document}

\maketitle

\begin{abstract}
IBM heeft voor zijn IBM Z Systems de deur opengezet naar vernieuwing door Java
en Python toegankelijk te maken op de mainframe. Hoewel deze talen ondersteund
worden, is het niet altijd even makkelijk om programma’s hierin te schrijven.
IBM Developer for z/OS is een stap in de goede richting door zijn functionaliteit om
in een bekend programma bestanden die opgeslagen zijn op de mainframe te openen
en aan te passen. Aangezien dit Eclipse based is, is er geen Python-interpreter
waardoor Python scripts moeten worden aangepast in een text editor.
Deze paper is een proof of concept voor het opstellen van Pydev als Python IDE in
IBM Developer for z/OS. Hierdoor kunnen Python developers efficiënter code schrijven
en aanpassen in de Unix System Services omgeving van de mainframe.
In dit onderzoek wordt er gewerkt met de Pydev-plug-in voor Eclipse en worden er
instructies gegeven over hoe dit ingesteld wordt. Ook wordt er onderzocht of deze
scripts uitgevoerd kunnen worden via IBM Developer for z/OS in de mainframeomgeving.
Om dit onderzoek te testen, zal er een Python API geschreven worden in IBM Developer
for z/OS. Deze zal uitgevoerd worden via dit programma op de Unix System
Services omgeving. Deze omgeving brengt andere problemen met zich mee die
opgelost zullen worden.
Het resultaat is een volledige proof of concept om een Python IDE in IBM Developer
for z/OS op te zetten. Verder volgt configuratie om een verbinding te maken
met de Unix System Services omgeving om Python-applicaties in uit te voeren. In
de testfase van dit onderzoek wordt een API geschreven en een stappenplan gegeven
over de benodigdheden om Python-applicaties in de gebruikte omgeving
uit te voeren. Dit is belangrijk voor de programmeurs van DNB die werken in dit
programma en voorlopig hun Python code moeten testen via een externe ssh connectie.
\end{abstract}

\begin{multicols}{2} % This is how many columns your poster will be broken into, a portrait poster is generally split into 2 columns

\section{Introductie}

In de huidige technologische wereld is het moeilijk
bij te houden welke mogelijkheden er zijn om
bepaalde projecten uit te voeren. Zo heb je altijd
nieuwe technieken die net iets efficiënter of
krachtiger zijn dan een ander. Door deze voortgaande
evolutie zullen individuen niet direct op
de hoogte zijn van recente ontwikkelingen, maar
vergeten ze ook de oudere toepassingen van een
bepaalde techniek. Dit is vooral merkbaar bij IBM
hun z Systems of mainframes die zo belangrijk,
maar zo snel vergeten worden in de IT wereld.
Hoewel het een zeer hoogwaardige technologie
is, zijn de technieken nog steeds zeer oud. Hier
wordt wel op ingezet door modernere talen zoals
Java en Python compatibel te maken met dit systeem.
Een veel gebruikte tool, aangeboden door
IBM, is IBM Developer for z/OS. Dit is een Eclipse
based programma geschreven in Java en kan een
connectie maken met de mainframe en zijn opgeslagen
bestanden. Dit maakt het eenvoudiger
voor gebruikers om verschillende bestanden
te openen en wijzigen maar heeft geen Python
interpreter waardoor het voor developers nogsteeds
niet zo efficiënt om Python programma’s mee aan te passen.

\section{Experimenten}

In dit onderzoek wordt er gewerkt met de Pydev plug-in voor Eclipse en worden er
stap voor stap instructies gegeven over hoe dit opgezet wordt in IBM Developer for
z/OS. Aangezien dit programma verschillend is van Eclipse, kunnen er problemen
opduiken die niet voorkomen in het standaard Eclipse-programma. \\ \\
De efficiëntie van Pydev wordt getest door een API te schrijven die doormiddel van de Z Open Automation Utilities 2 datasets zal lezen en 1 zal
schrijven. In deze testfase wordt er een oplossing geboden voor de meest voorkomende problemen bij het uitvoeren van Python-applicaties op de mainframe. \\ \\
Eens dit is opgezet, wordt er onderzocht of deze scripts kunnen worden uitgevoerd op de mainframe via IBM Developer for z/OS. De doelstelling van deze fase is het uitvoeren van de geschreven API in IBM hun programma. Als dit lukt, wordt er gekeken of er limitaties zijn vergeleken me een externe ssh connectie.  

\section{Gebruikte methodes}\\ \\

Om Pydev op te zetten in IBM Developer for z/OS, moest dit geïnstalleerd worden van hun website. Er werd gekozen voor een oudere versie omdat dit het meest compatibel was. \\
De geschreven applicatie maakt gebruik van de package FastAPI om inkomende verzoeken af te handelen. Dit moet geïnstalleerd worden via \url{https://pypi.org} samen met alle packages waarop FastAPI gebouwd is. \\
Deze API leest van een JSON bestand dus moet de package Jsonschema ook geïnstalleerd zijn. Dit kan via de Python AI toolkit for IBM z/OS.
Verder wordt er nog gebruik gemaakt van de Z Open Automation Utilities om verschillende soort acties uit te voeren op data opgeslagen op de mainframe. \\ \\
Om de API uit te voeren, wordt er een Python virtuele omgeving aangemaakt zodat er geen problemen zijn met administratierechten. Hierin worden de packages geïnstalleerd. De nodige omgevingsvariabelen worden ook correct opgezet in deze omgeving. De applicatie wordt opgeslagen in de IBM-1047 encoding wat EBCDIC is en dus geen ASCII. 

\section{Conclusies}\\ \\

Het gebruik van Pydev is zeker aan te raden in IDz om efficiënter Python applicaties te schrijven voor de mainframe. Hoewel het niet even performant is als andere editors zoals Visual Studio Code, is dit makkelijker voor Python developers om direct applicaties te maken in de mainframe omgeving. Hoewel het schrijven in IDz makkelijker is, is dit niet het geval voor het testen van deze applicaties. Dit is momenteel het meest performant in een externe ssh connectie.

\section{Toekomstig onderzoek}\\ \\

Dit onderzoek heeft duidelijk gemaakt dat Python compatibel is met de mainframe
en nu is er ook een mogelijkheid om applicaties in deze programmeertaal
direct te schrijven op de mainframe. Tools zoals ZOAU en de Python AI toolkit for
z/OS maken de mogelijkheden enorm groot waardoor er zeer complexe applicaties
geschreven kunnen worden. Door de afname van skills in COBOL en PL/1, is het interessant
om te onderzoeken of Python en Java de oudere programmeertalen voor
batch- en online-jobs volledig kunnen vervangen en of dit invloed zou hebben op
de snelheid waarmee deze jobs worden uitgevoerd.

\end{multicols}
\end{document}